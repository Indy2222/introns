\documentclass[a4paper,twoside]{ociamthesis}


%%%%% SELECT YOUR DRAFT OPTIONS
% Three options going on here; use in any combination. But remember to turn the
% first two off before generating a PDF to send to the printer!

% This adds a "DRAFT" footer to every normal page. (The first page of each
% chapter is not a "normal" page.)
% \fancyfoot[C]{\emph{DRAFT Printed on \today}}

% This highlights (in blue) corrections marked with (for words)
% \mccorrect{blah} or (for whole paragraphs)
% \begin{mccorrection} ... \end{mccorrection}. This can be useful for sending a
% PDF of your corrected thesis to your examiners for review. Turn it off, and
% the blue disappears.
\correctionstrue


%%%%% BIBLIOGRAPHY SETUP
% Note that your bibliography will require some tweaking depending on your
% department, preferred format, etc. The options included below are just very
% basic "sciencey" and "humanitiesey" options to get started. If you've not
% used LaTeX before, I recommend reading a little about biblatex/biber and
% getting started with it. If you're already a LaTeX pro and are used to natbib
% or something, modify as necessary. Either way, you'll have to choose and
% configure an appropriate bibliography format...

% The science-type option: numerical in-text citation with references in order
% of appearance.
\usepackage[style=numeric-comp, sorting=none, backend=biber, doi=false, isbn=false]{biblatex}
\newcommand*{\bibtitle}{References}

\usepackage{fancyvrb}
\usepackage{longtable}
\usepackage{pdfpages}
\usepackage{xcolor}

\usepackage{mathtools}
\DeclarePairedDelimiter\ceil{\lceil}{\rceil}

% This makes the bibliography left-aligned (not 'justified') and slightly smaller font.
\renewcommand*{\bibfont}{\raggedright\small}

% Change this to the name of your .bib file (usually exported from a citation
% manager like Zotero or EndNote).
\addbibresource{references.bib}


% Uncomment this if you want equation numbers per section (2.3.12), instead of
% per chapter (2.18):
% \numberwithin{equation}{subsection}


%%%%% THESIS / TITLE PAGE INFORMATION
% Everybody needs to complete the following:
\title{Automatic Intron Detection in Metagenomes Using Neural Networks}
\author{Bc. Martin Indra}
\college{Faculty of Electrical Engineering}


% Uncomment the following line if your degree also includes exams (eg most masters):
%\renewcommand{\submittedtext}{Submitted in partial completion of the}
% Your full degree name.  (But remember that DPhils aren't "in" anything.  They're just DPhils.)
% \degree{Master of Science}
% Term and year of submission, or date if your board requires (eg most masters)
\degreedate{Winter Term 2020/2021}


%%%%% YOUR OWN PERSONAL MACROS

%%%%% THE ACTUAL DOCUMENT STARTS HERE
\begin{document}

%%%%% CHOOSE YOUR LINE SPACING HERE
% This is the official option.  Use it for your submission copy and library copy:
\setlength{\textbaselineskip}{22pt plus2pt}
% This is closer spacing (about 1.5-spaced) that you might prefer for your personal copies:
%\setlength{\textbaselineskip}{18pt plus2pt minus1pt}

% You can set the spacing here for the roman-numbered pages (acknowledgements, table of contents, etc.)
\setlength{\frontmatterbaselineskip}{17pt plus1pt minus1pt}

% Leave this line alone; it gets things started for the real document.
\setlength{\baselineskip}{\textbaselineskip}


%%%%% CHOOSE YOUR SECTION NUMBERING DEPTH HERE
% You have two choices. First, how far down are sections numbered? (Below that,
% they're named but don't get numbers.) Second, what level of section appears
% in the table of contents? These don't have to match: you can have numbered
% sections that don't show up in the ToC, or unnumbered sections that do.
% Throughout, 0 = chapter; 1 = section; 2 = subsection; 3 = subsubsection, 4 =
% paragraph...

% The level that gets a number:
\setcounter{secnumdepth}{2}
% The level that shows up in the ToC:
\setcounter{tocdepth}{2}

% JEM: Pages are roman numbered from here, though page numbers are invisible
% until ToC. This is in keeping with most typesetting conventions.
\begin{romanpages}

% Title page is created here
\maketitle

%%%%% DEDICATION -- If you'd like one, un-comment the following.
%\begin{dedication}
%This thesis is dedicated to\\
%someone\\
%for some special reason\\
%\end{dedication}

%%%%% ACKNOWLEDGEMENTS
\begin{acknowledgements}
  I would like to thank my supervisor, doc. Ing. Jiří Kléma, Ph.D., for giving me
the opportunity to work on this interesting and valuable project and for all
his support and guidance.

I would also like to thank Ing. Anh Vu Le for providing me with a lot of
valuable data and insights into an existing intron detection pipeline based on
support vector machines.

In addition, I would like to thank my wife who supported me during the
difficult times.

\end{acknowledgements}

%%%%% ABSTRACT -- Nothing to do here except comment out if you don't want it.
\begin{abstract}
  \begin{minipage}[t]{0.48\textwidth}
  This work is concerned with the detection of introns in metagenomes with deep
  neural networks. Exact biological mechanisms of intron recognition and
  splicing are not fully known yet and their automated detection has remained
  unresolved.

  Detection and removal of introns from DNA sequences is important for the
  identification of genes in metagenomes and for searching for homologs among
  the known DNA sequences available in public databases. Gene prediction and
  the discovery of their homologs allows the identification of known and new
  species and their taxonomic classification.

  Two neural network models were developed as part of this thesis. The models'
  aim is the detection of intron starts and ends with the so-called donor and
  acceptor splice sites. The splice sites are later combined into candidate
  introns which are further filtered by a simple score-based overlap resolving
  algorithm.

  The work relates to an existing solution based on support vector machines
  (SVM). The resulting neural networks achieve better results than SVM and
  require more than order of magnitude less computational resources in order to
  process equally large genome.

  \vspace{\baselineskip}
  \textbf{Keywords:} fungi, fungal genomes, neural networks, itron detection,
  metagenome
\end{minipage}
\hspace{0.04\textwidth}
\begin{minipage}[t]{0.48\textwidth}
  Tato práce se zabývá detekcí intronů v metagenomech hub pomocí hlubokých
  neuronových sítí. Přesné biologické mechanizmy rozpoznávání a vyřezávání
  intronů nejsou zatím plně známy a jejich strojová detekce není považovaná za
  vyřešený problém.

  Rozpoznávání a vyřezávání intronů z DNA sekvencí je důležité pro identifikaci
  genů v metagenomech a hledání jejich homologií mezi známými DNA sekvencemi,
  které jsou dostupné ve veřejných databázích. Rozpoznání genů a nalezení
  jejich případných homologů umožňuje identifikaci jak již známých tak i nových
  druhů a jejich taxonomické zařazení.

  V rámci práce vznikly dva modely neuronových sítí, které detekují začátky a
  konce intronů, takzvaná donorová a akceptorová místa sestřihu. Detekovaná
  místa sestřihu jsou následně zkombinována do kandidátních intronů.
  Překrývající se kandidátní introny jsou poté odstraněny pomocí jednoduchého
  skórovacího algoritmu.

  Práce navazuje na existující řešení, které využívá metody podpůrných vektorů
  (SVM). Výsledné neuronové sítě dosahují lepších výsledků než SVM a to při
  více než desetinásobně nižším výpočetním čase na zpracování stejně obsáhlého
  genomu.

  \vspace{\baselineskip}
  \textbf{Klíčová slova:} genom hub, neuronové síťe, detekce intronů, metagenom
\end{minipage}

\end{abstract}

\begin{alwayssingle}
  \chapter*{Čestné prohlášení}

Prohlašuji, že jsem zadanou diplomovou práci zpracoval sám s přispěním
vedoucího práce a konzultanta a používal jsem pouze literaturu v práci
uvedenou. Dále prohlašuji, že nemám námitek proti půjčování nebo zveřejňování
mé diplomové práce nebo její části se souhlasem katedry.

\vspace{.5in}
\par\noindent\makebox[1.5in][c]{\dotfill}\hfill\makebox[3.0in][c]{\dotfill}
\par\noindent\makebox[1.5in][c]{datum}\hfill\makebox[3.0in][c]{podpis diplomanta}

  \thispagestyle{empty}
  \pagestyle{empty}
  \setlength{\baselineskip}{\frontmatterbaselineskip}
\end{alwayssingle}

\begin{alwayssingle}
  \includepdf[pages=-]{figures/assignment.pdf}
  \thispagestyle{empty}
  \pagestyle{empty}
  \setlength{\baselineskip}{\frontmatterbaselineskip}
\end{alwayssingle}

%%%%% MINI TABLES
% This lays the groundwork for per-chapter, mini tables of contents. Comment
% the following line (and remove \minitoc from the chapter files) if you don't
% want this. Un-comment either of the next two lines if you want a per-chapter
% list of figures or tables.
\dominitoc % include a mini table of contents
%\dominilof  % include a mini list of figures
%\dominilot  % include a mini list of tables

% This aligns the bottom of the text of each page. It generally makes things
% look better.
\flushbottom

% This is where the whole-document ToC appears:
\tableofcontents

\listoffigures
\mtcaddchapter
% \mtcaddchapter is needed when adding a non-chapter (but chapter-like) entity
% to avoid confusing minitoc

% Uncomment to generate a list of tables:
%\listoftables
%	\mtcaddchapter

%%%%% LIST OF ABBREVIATIONS
% This example includes a list of abbreviations. Look at text/abbreviations.tex
% to see how that file is formatted. The template can handle any kind of list
% though, so this might be a good place for a glossary, etc.
\begin{mclistof}{List of Abbreviations}{3.2cm}

\item[2D] Two dimensional, referring in this thesis to dimensions of data.

\item[DNA] Deoxyribonucleic acid

\item[RNA] Ribonucleic acid

\item[CDS] Coding sequence.

\item[UTR] Untranslated region

\item[SGD] Stochastic gradient descent

\item[ANN] Artificial Neural Network

\item[CNN] Convolutional Neural Network

\item[RNN] Recurrent Neural Network

\item[RCNN] Recurrent Convolutional Neural Network

\item[SVM] Support Vector Machine

\item[ReLU] Rectified Linear Unit

\item[LSTM] Long short-term memory unit in Recurrent Neural Network

\item[BLAST] Basic Local Alignment Search Tool

\item[ROC] Receiver Operating Characteristic

\item[AUC] Area under the curve

\end{mclistof}


% The Roman pages, like the Roman Empire, must come to its inevitable close.
\end{romanpages}


%%%%% CHAPTERS
\flushbottom

\chapter{\label{ch:intro}Introduction}

\minitoc

The kingdom of fungi is largely undiscovered and comprises 2.2--3.8 million
species \cite{hawksworth2017fungal}. Only a small fraction of this---around
100\,000 species---has been described in the literature
\cite{hawksworth2012global}. The kingdom plays a significant role in public
health, food biosecurity, and biodiversity \cite{microbiology2017stop}. It is,
therefore, important to develop new, computer-aided methods to bridge the gap
between what is known and what is unknown.

A gene is a sequence of DNA nucleotides encoding a gene product. A gene product
is a protein or form of RNA. In most organisms, the DNA or RNA sequence that
encodes the product is not continuous but interleaved with introns, which are
spliced out during gene expression. The remaining parts are called exons
\cite{alberts2018molecular}.

Metagenomics is the study of combined genetic material as retrieved directly
from a natural environment. The collection of genetic material retrieved from a
sample is known as metagenome, which, as opposed to a single genome, consist of
the DNA sequences of multiple organisms.

Gene homology refers to the similarity between the genetic sequences of two
genes due to shared ancestry. Two genes with similar DNA sequences and shared
ancestry are called homologous. The identification of homologous genes can be
used in recognizing individual organisms or kinships in the metagenome.

In a study, forest soil samples were extracted and their genetic material
sequenced. Traditional sequence alignment methods were utilized for gene
prediction within the metagenome, leading to the identification of a number of
fungal species within the sample. But the sample was significantly smaller than
expectations based on the estimates of a number of existing fungal species. The
present work seeks to improve the results of the gene prediction procedure. For
this, it identifies and excises introns from the metagenome and keeps only
evolutionarily more preserved exons in the sequences.

The goal of this work is to create a multistep software pipeline that
identifies introns in DNA sequences of potentially unknown fungal organisms.
The use of deep neural networks is elaborated. Their performances and
computational resource usage are compared with previously used methods based on
support vector machines and the potential of using the results in a larger gene
prediction pipeline.

The work uses a large preexisting annotated fungal DNA dataset. The solution is
subdivided into the following sub-goals:

\begin{itemize}
  \item detection of candidate splice site positions with the search for
    consensus dinucleotides,
  \item classification of candidate intron start positions, i.e. donor splice
    sites,
  \item classification of candidate intron end positions, i.e. acceptor splice
    sites,
  \item combining detected splice sites into intron start-end pairs based on
    their mutual distance,
  \item filtering the intron candidates resulting from the previous step by an
    overlap-resolving algorithm.
\end{itemize}

\section{Text Structure}

This chapter (\ref{ch:intro}) introduces the work by briefly explaining several
important biological concepts and puts the work into its context. Chapter
\ref{ch:background} explains biological phenomena related to the work, some
concepts and techniques from computer science--basics of recurrent and
convolutional neural networks (RCNN) and statistical measures used for
evaluation. Chapter \ref{ch:motivation} talks about further motivation for this
work, and Chapter \ref{ch:current-research} summarizes current research on the
topic. Chapter \ref{ch:data} contains an overview of the used data, the
statistics computed on the data, their source, and the used file types.

Chapter \ref{ch:rcnn} describes the recurrent convolutional networks used for
splice site classification. It also describes the architecture and training of
the networks with theoretical and practical reasoning. Chapter
\ref{ch:automation} talks about how data pre-processing and RCNN training and
evaluation were automated for fast and reproducible experimentation. And
finally, Chapter \ref{ch:evaluation} talks about the evaluation and achieved
results.

Chapter \ref{ch:conclusion} concludes the work and outlines possible further
work.

\chapter{\label{ch:background}Background}

\minitoc

The nucleic DNA molecules or chromosomes of Eukaryotic organisms play many
roles in cell biology. Different positions (loci) in a chromosome play
different, often overlapping and complex, roles \cite{pennisi2007dna}. DNA
functions and its structure are not yet fully understood
\cite{slijepcevic2018genome}.

The previously mentioned complexity opens up the possibility of novel
approaches like state-of-the-art machine learning techniques in order to study
the DNA structure, its roles and functions, and the detection of already known
parts in new DNA sequences. For example, it has been shown that feed-forward
neural networks can represent a wide variety of functions
\cite{cybenko1989approximation}. The versatility and effectiveness of
artificial neural networks has been practically demonstrated in many fields.
Examples of this are general game play \cite{silver2016mastering}, face
recognition \cite{sun2015deepid3}, speech recognition
\cite{xiong2016achieving}, or medical image classification
\cite{bychkov2018deep}, among others.

\section{\label{ch:background:dna-rna}DNA and RNA Structure}

Polynucleotide is a linear chain of up to 20 different nucleotide monomers
joined by the phosphodiester (covalent) bond
\cite[p.~308,p.~347]{king2006dictionary}. Deoxyribonucleic acid (DNA) and
ribonucleic acid (RNA) are two types of polynucleotides that are abundant in
natural life. Both have important biological functions. DNA is composed of
individual nucleotides Adenine (A), Thymine (T), Cytosine (C), and Guanine (G)
\cite[p.~4,~p.~20,~p.~107]{pollard2016cell}. RNA is made from nucleotides
Adenine (A), Uracil (U), Cytosine (C), and Guanine (G)
\cite[p.~11]{jankowski1996clinical}. RNA molecules first appeared around four
billion years ago as a first form of life \cite[p.~412]{king2006dictionary}.

The DNA molecule is directional due to asymmetry of individual nucleotides
\cite[p.~42]{pollard2016cell}. See Figure \ref{fig:intro:nucleotide}. Based on
The chemical convention of naming carbon atoms in the nucleotide sugar-ring,
one side of DNA is called 5\textquotesingle{}-end and the other side
3\textquotesingle{}-end. DNA and RNA are synthesized in the 5\textquotesingle{}
to 3\textquotesingle{} direction \cite[p.~167,~p.~728]{pollard2016cell}. When
referring to relative positions in a DNA sequence, upstream and downstream
refer to the 5\textquotesingle{} and the 3\textquotesingle{} directions
respectively. Similarly, by convention, DNA sequences are usually written and
stored in the 5\textquotesingle{} to 3\textquotesingle{} direction, unless
explicitly stated or needed otherwise.

Chromosomes are enormous DNA molecules which encode the majority of genetic
information in fungi and other kingdoms of species. DNA in chromosomes consists
of two coiled chains of polynucleotide strands forming a double helix.

\begin{figure}
  \centering
  \includegraphics[width=0.3\textwidth]{figures/nucleotide.pdf}
  \caption{Molecular structure of a nucleotide \cite{nucleotide-img}}
  \label{fig:intro:nucleotide}
\end{figure}

Different parts of a chromosome have different functions in a cell. Within
chromosomes are continuous parts which form genes, which are nucleotide
sequences that encode gene products---either RNA or protein. Genes have a
complex structure that contains regulatory sequences, exons, introns, and other
areas. The regulatory sequences of a gene, located at the extremities of the
gene, contain a promoter at the 5\textquotesingle{} side and a terminator at
the 3\textquotesingle{} side of the gene. The promoter and terminator mark the
beginning and end of the transcribed region of the gene respectively.

In eucaryotic organisms, gene transcription forms a primary transcript,
alternatively called precursor mRNA or Pre-mRNA, which consists of exons and
introns and lacks 5\textquotesingle{} cap and poly(A) tail. 5\textquotesingle{}
cap and poly(A) tail are added at later stages of gene expression
\cite{cooper2000cell}. Afterward, introns are spliced out during the
post-transcriptional modification and the remaining exons form a final mature
mRNA that, in some cases, encodes protein. Both ends of mRNA contain
untranslated regions (UTR) 5\textquotesingle{}-UTR and 3\textquotesingle{}-UTR
enclosing the final protein coding region \cite{shafee2017eukaryotic}. The gene
structure and processing are illustrated in Figure
\ref{fig:intro:gene-structure}.

\begin{figure}
  \centering
  \includegraphics[width=0.9\textwidth]{figures/gene-structure.pdf}
  \caption{Gene structure and processing \cite{shafee2017eukaryotic}}
  \label{fig:intro:gene-structure}
\end{figure}

Within introns, the donor splice site, branch point, and the acceptor splice
site are used for splicing. The donor splice site lies at the
5\textquotesingle{}-end of the intron, the acceptor splice site lies at the
3\textquotesingle{}-end of the intron, and the branch point lies 18--40
nucleotides upstream from the acceptor site \cite{clancy2008rna}. In the great
majority of the cases, the 5\textquotesingle{}-end of an intron starts with
highly preserved \Verb_GU_ nucleotides and the 3\textquotesingle{}-end of an
intron ends with \Verb_AG_ nucleotides. The highly preserved dinucleotides,
which make first two symbols in an intron (at 5\textquotesingle{}-end) and last
two symbols in an intron (at 3\textquotesingle{}-end), are commonly referred to
as consensus dinucleotides. The branch point always contains adenine, but it is
otherwise more variable \cite{clancy2008rna}. Cell's splicing machinery is
called spliceosomes and removes introns in a sequence of complex steps. The
exact end-to-end mechanisms are not yet fully understood \cite{clancy2008rna}.

\section{\label{ch:background:alignment}Biological Sequence Alignment and Search}

Sequence alignment is an algorithm which searches for the best scoring
alignment of two or more sequences of symbols by inserting gaps into any of the
sequences. The resulting alignment contains indels (insertions and deletions),
mismatches, and matches. The alignments are usually scored by summing penalties
for indels with (mis)match scores at other positions. The (mis)match scores are
taken from a substitution matrix and the gaps are frequently scored with a
gap-open penalty summed up with a penalty proportional to the gap length. An
example pairwise alignment of two sequences is given in Figure
\ref{fig:background:global-alignment}.

\begin{figure}
\centering
\begin{BVerbatim}[baselinestretch=0.75]
A-TATCATGA
AGTA-CATGG
\end{BVerbatim}
\caption{Global sequence alignment example}
\label{fig:background:global-alignment}
\end{figure}

Two types of alignments exist: global and local. A global alignment produces
sequences of equal length, while a local alignment produces an alignment only
of parts of the sequences. See Figure \ref{fig:background:local-alignment} with
an example of a local pairwise alignment.

\begin{figure}
\centering
\begin{BVerbatim}[baselinestretch=0.75]
ATTATCATGA
  TA-CA
\end{BVerbatim}
\caption{An example of a sequence (second row) locally aligned to a larger
  sequence (first row)}
\label{fig:background:local-alignment}
\end{figure}

There exist exact algorithms like the Needleman-Wunsch algorithm for global
alignments \cite{needleman1970general} and the Smith-Waterman algorithm for
local alignments \cite{smith1981identification}. The exact algorithms have
large asymptotic time and memory complexity. An optimized version of the
Needleman-Wunsch algorithm has $\mathcal{O}(mn / \log(n))$ asymptotic time
complexity, where $m$ and $n$ are lengths of the sequences
\cite[p.~35]{sung2009algorithms}. The asymptotic time complexity of the
Smith-Waterman algorithm is $\mathcal{O}(mn)$ \cite[p.~40]{sung2009algorithms}.
This makes their usage for search in large databases unfeasible because their
asymptotic time has to be further multiplied by the number of database entries.

For reasons of efficiency, heuristic sequence alignment algorithms are in wide
use. These algorithms do not guarantee optimal results but have a much shorter
execution time and smaller memory usage. Among the heuristic algorithms is
BLAST (basic local alignment search tool), one of the most used algorithms for
sequence searching \cite{casey2005blast}.

E-value is the expected number of coincidental matches in the database of a
given size with the match score equal or greater than the found score
\cite[p.~119]{sung2009algorithms}. In other words, a large E-value signifies a
high probability that a match is only coincidental.

When searching for a sequence in a database, a local alignment, such as BLAST,
is performed against all available sequences. All matches with a high enough
score giving a sufficiently small E-value are reported as results.

\section{Neural Networks}

The artificial neural network is a network of interconnected artificial
neurons. The activation $y_q$ (output) of a simple artificial neuron, $q$, can
be described with Formula \ref{eq:intro:neuron-activation}, where $g:
\mathbb{R} \rightarrow \mathbb{R}$ is the neuron activation function, $x_i$ is
either one of the neural network inputs or the activation of the preceding
neuron $i$, $w_{iq} \in \mathbb{R}$ is the weight of the connection from neuron
$i$ to neuron $q$, and $b_q \in \mathbb{R}$ is the bias of neuron $q$. The
weights and biases are learned during ANN training.

\begin{equation}
  y_q = g\left(\sum(x_i \cdot w_{iq}) - b_q\right)
  \label{eq:intro:neuron-activation}
\end{equation}

The architecture of a neural network consists of the connections between
individual neurons and activation functions, which are specified as
meta-parameters \cite[p.~193]{goodfellow2016deep}. Most neural networks could
be organized into layers. The neurons in each layer connect only to neurons
from preceding layers \cite[p.~193]{goodfellow2016deep}.

A neural network is considered deep if it consists of more than three levels of
compositions of non-linear operations \cite[p.~6]{bengio2009learning}.
Therefore, a neural network structured into layers is deep if it comprises more
than one hidden layer.

\begin{figure}
  \centering
  \includegraphics[width=0.3\textwidth]{figures/neural-network.pdf}
  \caption{Feed forward artificial neural network \cite{feed-forward-img}}
  \label{fig:intro:neural-network}
\end{figure}

Training algorithms, often called neural network optimizers, try to minimize
the value of the loss function on training data. The loss function measures how
well the network performs in each individual training sample, for example, the
difference between truth and prediction.

Neural networks are usually trained with a variation of stochastic gradient
descent (SGD) or a derived algorithm \cite[p.~149]{goodfellow2016deep}. SGD
repeatedly and randomly selects several training samples called a mini-batch
and computes the loss function gradient with respect to trained parameters
(i.e. weights and biases) with the back propagation algorithm
\cite[p.~149]{goodfellow2016deep}. Backpropagation is an algorithm which
computes the gradient of a loss function with respect to network parameters
with application of the chain rule \cite[p.~201]{goodfellow2016deep}. A
multiple, which is called the learning rate, of the calculated gradient is then
subtracted from the parameters \cite[p.~150]{goodfellow2016deep}.

A common pattern in the architecture of binary classification neural networks
is to have a single output neuron, with an appropriate activation function and
a discrimination threshold on output neuron activation.

Convolutional neural networks (CNNs) are a class of neural networks that
contain one or more convolution layers (see Figure \ref{fig:intro:cnn}).
Convolution layers contain neurons with shared weights and a limited perceptive
field; the weights of the neurons are shift invariant
\cite[p.~326]{goodfellow2016deep}. Thanks to its features, CNNs greatly reduce
a number of learnable parameters and are therefore easier to train and less
prone to overfitting \cite[p.~339]{goodfellow2016deep}.

\begin{figure}
  \centering
  \includegraphics[width=0.4\textwidth]{figures/cnn.png}
  \caption{Convolutional layer (top) over preceding layer (bottom)
    \cite{dumoulin2016guide}}
  \label{fig:intro:cnn}
\end{figure}

Formula \ref{eq:intro:cnn} provides the activation of neurons in a 2D
convolutional layer, where $y_{i, j}$ is the activation of the neuron at
position $(i, j)$ in a 2D grid of neurons, $g: \mathbb{R} \rightarrow
\mathbb{R}$ is the activation function, $k_{m, n} \in \mathbb{R}$ is a shared
weight from kernel $k \in \mathbb{R}^{M \times N}$, $x_{i + m, j + n}$ is the
output of neuron $(i + m, j + n)$ from the preceding 2D layer, and $b \in
\mathbb{R}$ is the shared bias \cite[p.~328]{goodfellow2016deep}.

\begin{equation}
  y_{i, j} = g\left(\sum_m \sum_n k_{m, n} \cdot x_{i + m, j + n} - b\right)
  \label{eq:intro:cnn}
\end{equation}

Recurrent neural networks (RNNs) are a family of neural networks where the
connections between nodes form a directed graph along a sequence
\cite[p.~368]{goodfellow2016deep}. In other words, the computational graph of
these network contains loops that unfold over a time variable. This
architecture allows for the processing of possibly arbitrarily long sequences
of data \cite[p.~367]{goodfellow2016deep}.

\section{\label{ch:background:evaluation}Evaluation}

The four following basic metrics can be estimated for any binary classification
algorithm with respect to truth data:

\begin{itemize}
  \item true positive rate ($\mathit{TPR}$) -- probability of a positive sample
    being classified as positive by the classification algorithm,
  \item true negative rate ($\mathit{TNR}$) -- probability of a negative sample
    being classified as negative,
  \item false positive rate ($\mathit{FPR}$) -- probability of a negative
    sample being classified as positive,
  \item false negative rate ($\mathit{FNR}$) -- probability of a positive
    sample being classified as negative.
\end{itemize}

These constants are estimated with formulas $\mathit{TPR} =
\frac{\mathit{TP}}{P}$, $\mathit{TNR} = \frac{\mathit{TN}}{N}$, $\mathit{FPR} =
\frac{\mathit{FP}}{N}$ and $\mathit{FNR} = \frac{\mathit{FN}}{P}$, where
$\mathit{TP}$, $\mathit{TN}$, $\mathit{FP}$, $\mathit{FN}$, $P$, and $N$ are
the number of true positives, true negatives, false positive, false negatives,
the number of positive samples, and the number of negative samples,
respectively, and obtained by running the algorithm on a test dataset.

Many other metrics could be derived from these constants.

\subsection{Accuracy}

The accuracy of an algorithm is the probability of a sample being classified
correctly, given by Formula \ref{eq:intro:accuracy}, where $p$ is the prior
probability of the positive class.

\begin{equation}
  \mathit{Accuracy} = \mathit{TPR} \cdot p + \mathit{TNR} \cdot (1 - p)
  \label{eq:intro:accuracy}
\end{equation}

This metric is of limited use in case of highly unbalanced data. For example,
an algorithm classifying all samples as negative would have 99\% accuracy on
data with 99\% of the samples being negative.

\subsection{Precision and Recall}

Precision is a fraction of the true positive classifications in all positive
classifications given by Formula \ref{eq:intro:precision}, where $p$ is the
prior probability of the positive class.

\begin{equation}
  \mathit{Precision} = \frac{\mathit{TPR} \cdot p}{\mathit{TPR} \cdot p +
    \mathit{FPR} \cdot (1 - p)}
  \label{eq:intro:precision}
\end{equation}

Recall is a fraction of the positive samples classified as positive. Recall is
independent of class prior probabilities. See Formula \ref{eq:intro:recall}.

\begin{equation}
  \mathit{Recall} = \frac{\mathit{TPR}}{\mathit{TPR} + \mathit{FNR}} =
  \mathit{TPR}
  \label{eq:intro:recall}
\end{equation}

Both precision and recall might be calculated directly from $\mathit{TP}$,
$\mathit{FP}$, $\mathit{TN}$, $\mathit{FN}$ counts measured on a test dataset.
In such a scenario, calculated precision would differ from
\ref{eq:intro:precision} if a fraction of positive sample was different from
positive class prior $p$.

\subsection{Area Under Curve}

The area under curve (AUC) is an area under a receiver characteristic curve
(ROC). An ROC curve gives dependence of a true positive rate on the false
positive rate obtained on the different threshold settings of a binary
classification algorithm.

\chapter{\label{ch:motivation}Motivation}

\minitoc

DNA sequencing costs are sharply decreasing \cite{NHGRI_sequencing_costs}.
GenBank is a genetic sequence database of all publicly available DNA sequences
\cite{benson2012genbank}. The number of bases in the GenBank database has
doubled approximately every 18 months from 1982 to present (2019), and its
database consists of over 366 billion bases \cite{genbank_release_notes}. The
consequence of ever-cheaper DNA sequencing and exponential sequence database
growth is the need for an increased capacity for DNA data handling and
analysis. An automated and scalable solution to DNA annotation would partially
fulfill this need.

Fungal and bacterial organisms play an important role in various ecosystems
including the floor and soil of forests
\cite{christensen2005wood}\cite{bani2018role}. Some fungal species are capable
of decomposing cellulose and various biopolymers and are involved in the
decomposition of deadwood and litter. The decomposition goes in stages, and in
each stage different organisms with varying diversity contribute to the process
\cite{bani2018role}. Nontrivial dependencies of different fungal organisms and
decomposition stages were identified and a lot of the relationships are yet to
be known \cite{bani2018role}. Decomposed wood and litter is important for some
plant species \cite{bani2018role}. The presence of wood inhabiting fungi might
be a good indicator of overall forest biodiversity \cite{christensen2005wood}.

Not only for the previously mentioned reasons, the ability to identify fungal
species in metagenomes recovered from soil and other environmental samples is a
potent tool in the biological study of forests and other ecosystems. The task
of fungal species identification in a metagenome is approached by searching for
homologous genes in the DNA sequences by applying sequence alignment algorithms
to all known protein and gene sequence databases. The search for homologs does
not only allow the identification of known species, but also searches for
novel, evolutionary related species.

To achieve good search results, we detect and remove introns in long DNA
sequence scaffolds before the search. This should improve protein sequence
match scores because introns are spliced before RNA to protein translation. A
reason for the improvement is that the match score of alignment algorithms for
homologous DNA sequences is the highest in most preserved parts of such DNA
sequences---i.e. exons. The functional regions of DNA sequences tend to be more
highly conserved than the remaining parts of the DNA
\cite{morgenstern2002exon}. See Section \ref{ch:background:alignment} for more
information on sequence alignment.

\chapter{\label{ch:current-research}Current Research}

\minitoc

Research on automated splice site detection goes back to at least 1987. For
example, the work \cite{shapiro1987rna} used the probability of occurrence of
particular nucleic bases at particular positions for splice site detection.

The hidden Markov model and the AdaBoost classifier to detect an intron splice
site were used in \cite{pashaei2016novel}. The hidden Markov model was also
used in \cite{goel2015improved} for sequence pre-processing and the support
vector machine for intron splice site detection. A multilayered recurrent
neural network applied on triplets from original sequence is described in
\cite{sarkar2019splice}. A convolutional neural network and its interpretation
is described in \cite{zuallaert2018splicerover}, which reports that the network
is most sensitive to nucleotides close to a splice site.

Good performance of recurrent convolutional neural networks in detecting the
protein binding motifs in DNA sequences is reported in
\cite{hassanzadeh2016deeperbind}. The algorithm is named DeeperBind. DeeperBind
encodes DNA sequences with one-hot-encoding. The network consists of one
convolutional layer, followed by two LSTM layers, and finally, the LSTM output
is connected to two fully connected layers with dropout. The convolutional
layers are not followed by max pooling layers common in CNN architectures.
Strides of convolution is set to one. The paper reports that the following LSTM
layers are able to deal with increased redundancy produced with convolutional
layers with stride one and without max pooling.

The paper \cite{lee2015dna} takes a different approach to RNA sequence
encoding. The authors report diminished generalization when learning
one-hot-encoded sequences and overcoming this issue by embedding each
nucleotide to a four-dimensional dense vector. The input is followed by two RNN
layers whose output is fed to a fully connected layer with dropout. The paper
compares the performances of ReLU, LSTM, and GRU recurrent units. It is
concluded that the addition of more RNN layers does not further improve the
performance and that LSTM units show the best performance.

The use of recurrent convolutional neural networks for splice site detection in
the human genome is described in \cite{naito2018human}. Two convolutional
layers, each succeeded by a max-pooling layer, are followed by a bidirectional
LSTM recurrent layer. The output of the RNN layer is fed to two fully connected
layers with soft-max at the final output. Dropout is applied on the outputs of
all layers. One-hot-encoding is used on the inputs. The paper reports improved
performance of the network with the RNN layer as compared with the same network
without that layer.

Past research is mostly based on classical methods; the use of deep neural
networks for splice site detection is rare. Some methods show almost perfect
performance \cite{sarkar2019splice}. However, these methods are usually trained
and tested on a single or a few organisms---for example, Homo Sapiens---and
training-testing data is split on the level of individual samples as opposed to
organism or higher taxonomic ranks. The ability of these algorithms to
generalize to the DNA sequences of species not present in training data is
largely unreported.

It has been described that the splice site recognition process is complex and
tissue-specific in humans \cite{pineda2018most}. This complexity may limit the
maximum achievable accuracy of splice site detection algorithms that work with
DNA sequences only on out-of-sample organisms.

Performance of convolutional neural networks on various classification problems
of DNA sequences was evaluated in \cite{nguyen2016dna}. The neural networks
were evaluated on 12 different datasets and 3 different classification tasks:

\begin{itemize}
 \item classification of sequences into histone-wrapping and
   non-histone-wrapping sequences,
 \item classification of sequences into three groups: sequences containing a
   donor splice site, sequences containing an acceptor splice site and other
   sequences,
 \item classification of DNA sequences into those containing nucleotides of a
   gene promoter and other sequences.
\end{itemize}

The input to the neural networks is a sequence of one-hot-encoded 3-mers. The
networks consist of two convolutional layers followed by a fully connected
layer with dropout of $0.5$. Softmax output layer is used. Modest accuracy is
reported in the paper. Accuracy over 96\% is reported for classification of the
splice sites. The splice site classification task described in the paper is
different to the one in the thesis because the paper does not solve the
detection of exact splice site position.

Convolutional neural networks are used in work \cite{senior2020improved} for
predicting the discrete probability distribution of distances and torsions of
amino acid residues based on pre-processed protein sequences. The neural
network is structurally similar to the networks used in image-recognition
tasks. The predicted distances were used as a seed for a gradient-descent
algorithm that predicts the three-dimensional structure of a protein by
minimizing the potential. The paper reports the state-of-the-art performance
and demonstrates the ability of deep neural networks to predict the physical
shapes of biological polymers.

The specifics of fungal DNA are exploited in the thesis. The ability of various
neural networks to generalize across various taxonomic ranks is studied. The
sheer amount of available annotated data enables the use of more complex neural
networks, which, in turn, can learn more subtle or complex features.

\chapter{\label{ch:data}Data}

\minitoc

Source DNA sequence data and its annotations were downloaded from the Joint
Genome Institute\footnote{https://jgi.doe.gov/fungi}
\cite{grigoriev2014mycocosm}. The data contains FASTA files with the DNA
sequence scaffolds of individual organisms and GFF files with DNA feature
annotations.

As part of this work, all data was uploaded to Google Cloud Storage in a
compressed form, and an automated download and extraction script was created.

\section{\label{ch:data:file-formats}File Formats}

A FASTA file is a text-based representation of DNA sequences. Such a file
contains header lines beginning with character \Verb_>_, followed by an
identifier of the sequence on successive lines. The sequence itself consists of
characters \Verb_ATCGN_ that represent adenine, thymine, cytosine, guanine, and
``any character'' respectively. In the data, each sequence within a FASTA file
represents a scaffold.

The sequences in FASTA files represent scaffolds. A scaffold links together a
non-contiguous series of genomic sequences, comprising sequences separated by
gaps of known length. The linked sequences are typically contiguous sequences
corresponding to read overlaps \cite{ison2013edam}.

The following sample is the first four lines of the FASTA file with the DNA
sequence of the organism Verticillium dahliae.

\begin{Verbatim}[fontsize=\small]
>Supercontig_1.1
AGTATCATGAAGGAAGAACAAGTTGAGGGACATAATTACCTGGGGTGCGGCGCTTACAAGTAAGGGTCGC
TGGGACATCGACCTGGAGGAGGAGAATCATGTAACGCCCCAGCCCGGTCGTCACCAGGACACCAGGCAGG
ACACCCCGCAGGCGATCGGACGCGCCGCACGGACCCACAGGATCACTCACGTGACCGTGACCAGATCACG
\end{Verbatim}

General Feature Format (GFF) is a file with DNA feature annotations. The file
format has three versions. The latest version 3 is used in the data. GFF is a
text-based file in which each line represents a feature annotation. An
annotation consists of nine tab-delimited attributes \cite{gff}:

\begin{itemize}
\item sequence -- name of the sequence, scaffold in our case, where the
  sequence is located,
\item source -- source of the feature, for example, name an institution,
\item feature -- type of the feature, only feature type \Verb_exon_ are used in
  this work,
\item start -- 1-based, inclusive offset of start of the feature,
\item end -- 1-based, inclusive offset of end of the feature,
\item score -- confidence in validity of the feature,
\item strand -- DNA strand where the feature is located. It is one of \Verb_+_,
  \Verb_-_ or \Verb_._ for positive, negative, and undetermined respectively,
\item phase -- phase of CDS features which is always one of 0, 1 or 2,
\item attributes -- additional feature attributes. Format of this field is not
  generally determined. In GFF files used in this work, it contains
  colon-delimited list of attributes, where each is space-delimited attribute
  name and value.
\end{itemize}

The following sample is the first four lines of a GFF file with annotations of
the DNA sequence of the organism Verticillium dahliae.

\begin{Verbatim}[fontsize=\scriptsize]
Supercontig_1   JGI exon    76  572 .   +   .   name "VDBG_00001T0"; transcriptId 224
Supercontig_1   JGI CDS 406 572 .   +   0   name "VDBG_00001T0"; proteinId 1; exonNumber 1
Supercontig_1   JGI start_codon 406 408 .   +   0   name "VDBG_00001T0"
Supercontig_1   JGI exon    631 1621    .   +   .   name "VDBG_00001T0"; transcriptId 224
\end{Verbatim}

\section{\label{ch:data:taxonomy}Taxonomy}

The data consists of eight different phyla, but $89.8\%$ of the organisms (774
of 862) are from Ascomycota and Basidiomycota phyla which make the Dikarya
subkingdom. See Table \ref{table:data:num-organisms} with the number of
organisms per phylum and Figure \ref{fig:data:tree} with a phylogenetic tree of
available fungi.

\begin{table}
  \begin{center}
    \begin{tabular}{ | l | c | }
      \hline
      \textbf{Phylum} & \textbf{Number of Organisms} \\
      \hline
      Ascomycota & 479 \\
      Basidiomycota & 295 \\
      Blastocladiomycota & 4 \\
      Cryptomycota & 1 \\
      Chytridiomycota & 21 \\
      Microsporidia & 8 \\
      Mucoromycota & 38 \\
      Zoopagomycota & 16 \\
      \hline
    \end{tabular}
  \end{center}
  \caption{\label{table:data:num-organisms}Phyla in the dataset}
\end{table}

\begin{figure}
  \centering
  \includegraphics[width=\textwidth]{figures/taxonomy-tree.png}
  \caption{Phylogenetic tree of fungi with sequenced genomes as displayed on
    the website of Joint Genome Institute \cite{grigoriev2014mycocosm}}
  \label{fig:data:tree}
\end{figure}

\section{\label{ch:data:stats}Data Statistics}

\begin{table}
  \begin{center}
    \begin{tabular}{ | l | c | }
      \hline
      \textbf{Taxonomy Rank} & \textbf{Number of Taxa} \\
      \hline
      Phylum & 8 \\
      Class & 46 \\
      Order & 124 \\
      Family & 307 \\
      Genus & 566 \\
      Species & 862 \\
      \hline
    \end{tabular}
  \end{center}
  \caption{Number of taxa on different taxonomy ranks}
\end{table}

In total, 16\,067\,492 introns, distributed over 5\,557\,272 individual genes,
were identified from exon annotations on all 862 organisms in the dataset. This
number includes only introns inside coding sequences, see Section
\ref{ch:automation:features}. Also see Section \ref{ch:automation:overview}
which includes the definition of gene used in this thesis.

The cumulative length of all genes is 8\,396\,757\,084 or 8\,371\,133\,149,
with overlaps counted only once. The cumulative length of all introns is
1\,296\,667\,885 or 1\,293\,833\,629, with counting overlaps only once or
roughly 15\% of genes.

The mean gene length is $1510.9$ nucleotides and the mean intron length is
$80.7$ nucleotides. As much as $93.5\%$ of the introns are 150 nucleotides long
or less with a distinct peek in length distribution around 50 nucleotides. See
Figure \ref{fig:data:intron-len-dist} with intron length distribution and
Figure \ref{fig:data:intron-len-box-plot} that depicts a box plot of intron
lengths on all available phyla.

\begin{figure}
  \centering
  \includegraphics[width=\textwidth]{figures/intron-length-hist.pdf}
  \caption{Distribution of intron length over all organisms in the dataset}
  \label{fig:data:intron-len-dist}
\end{figure}

In this work, only donors and acceptors with consensus dinucleotides were used,
see Section \ref{ch:rcnn:candidates} for more information on this topic.

In total, 15\,792\,942 donors and 15\,828\,432 acceptors were identified in the
data. All occurrences of \Verb_GT_ for donors and \Verb_AG_ for acceptors at
positions different from the positions of true splice sites were considered as
false donors and false acceptors. Moreover, 411\,393\,734 false donors and
498\,273\,594 false acceptors were found inside genes---this gives the
frequency of one false donor per $20.3$ nucleotides and one false acceptor per
$16.8$ nucleotides. False donors are $26.0$ times more frequent than true
donors, and false acceptors are $31.2$ times more frequent than true acceptors
in exon-intron areas.

\begin{figure}
  \centering
  \includegraphics[width=\textwidth]{figures/intron-length-box-plot.pdf}
  \caption{Box plot with intron lengths on all phyla. Whiskers extend to 5\%
    and 95\% percentiles.}
  \label{fig:data:intron-len-box-plot}
\end{figure}

\chapter{\label{ch:rcnn}Recurrent Convolutional Neural Networks}

\minitoc

This work aims to develop a method based on deep neural networks which would
detect and remove introns from the DNA sequences of various fungal organisms.

Removal of introns is done in these steps:

\begin{enumerate}
  \item all candidate donor and acceptor splice sites are identified based on
    consensus dinucleotides,
  \item these candidate splice sites are classified as true and false splice
    sites with separate donor and acceptor recurrent neural networks,
  \item positive classifications are combined to form candidate introns,
  \item candidate introns are assigned a score equal to the multiple of the
    respective donor and acceptor splice site model confidence,
  \item and overlapping candidate introns with non-highest score are filtered
    out.
\end{enumerate}

Section \ref{ch:rcnn:overview} gives an overview of the neural networks used
and the overall pipeline. Section \ref{ch:rcnn:candidates} talks about the
selection of candidate splice sites---i.e. locations in source DNA sequences
(from a metagenome) which are considered as potential donors and acceptors.
These locations are then classified with the splice site networks. Section
\ref{ch:rcnn:encoding} describes how the DNA sequences are encoded in matrices
so that they could be used as inputs to the neural networks. Section
\ref{ch:rcnn:architecture} talks about the architecture and other aspects of
the splice site classification neural network. Section \ref{ch:rcnn:criteria}
defines the criteria used during network selection. Finally, Section
\ref{ch:rcnn:splice-site-training} describes the training of the neural
networks.

\section{\label{ch:rcnn:overview}Overview}

All critical sections in the aforementioned intron detection and deletion of
the pipeline---i.e. two neural networks---are created as part of this work. An
existing pipeline from work \cite{barucic} (which originally used the support
vector machines in the critical section) was adapted to obtain a complete
pipeline.

Two separately trained neural networks with equal architecture were used for
the classification of donor and acceptor splice sites within original DNA
sequences (i.e. in full DNA sequence and before transcription to RNA). To
accompany the complexity of all regulatory and other sequences involved in
intron splicing, convolutional layers were utilized in target neural networks.
Recurrent layers with long short-term memory (LSMT) units were used to enable
the detection of partially shift invariant inter-dependencies and redundancies
between non-adjacent parts of the intron sequences.

The neural networks were trained to distinguish between true and false splice
sites. The input to the neural network is a window spanning some distance to
both sides around the candidate splice site. The window size has been selected
to be large enough so that the network can detect all important features within
the DNA sequence. The output of the network is the confidence of the input
sample being a true splice site.

The pipeline reported in \cite{barucic} requires a third model which further
filters candidate introns. The need for this additional step was motivated by
the high false positive rate of splice site classification models. In this
work, the additional filtering step was assessed as counterproductive because
splice site classification models based on neural networks have a smaller false
positive rate and further filtering of introns would impact the overall intron
recall. See Section \ref{ch:evaluation:comparison} for evaluation metrics.

\section{\label{ch:rcnn:candidates}Candidate Splice Site Selection}

Only splice sites with the consensus dinucleotides were used during the
training, evaluation, and testing of the networks. These are \Verb_GT_ for
donor and \Verb_AG_ for acceptor splice sites. This decision was made because
non-consensus splice sites are rare, and therefore, there was not enough
training and evaluation data to be included. In fact, its inclusion would
drastically increase the number of false positive classifications. It omission
has only a minuscule negative effect on the number of false negative
classifications. See Section \ref{ch:data:stats} which contains more related
statistics.

False splice sites included in training, validation, and test datasets are
selected only from the gene areas in source DNA sequences. See Section
\ref{ch:automation:overview} for the definition of ``gene'' used in this
thesis. Filtering to genes is motivated by the fact that intra-genetic DNA does
not play an important role during sequence alignment and homologous gene search
in the larger gene prediction pipeline. Therefore, removal of presumed introns
from these intra-genetic areas should not have a large impact on overall
results. Only the ability to recognize false splice site within genetic areas
was assumed to play an important role. Furthermore, sequences which would be
biologically processed as introns might be present in intra-genetic areas and
are therefore labeled as negative examples. This would lead to confusion of the
trained network as well as a worse performance.

\section{\label{ch:rcnn:encoding}Sequence Encoding}

Splice site classification neural networks were trained to map an input
sequence window to a value between 0 and 1. The sequence is encoded as a matrix
of dimensions $N \times 5$, where the $N$ rows represent relative positions
along the DNA sequence and the columns represent one-hot-encoded nucleotides.
Various input window sizes $N = N_{upstream} + N_{downstream}$ were evaluated,
see Table \ref{table:rcnn:win-size-donor} and Table
\ref{table:rcnn:win-size-acceptor}. Each nucleotide is encoded as a
5-dimensional vector, with value $1.0$ at the dimension of the represented
nucleotide and values $0.0$ at other dimensions. The first four dimensions
represent the nucleotides adenine (A), thymine (T), cytosine (C), and guanine
(G). The fifth dimension represents ``any symbol'' (usually missing or corrupt
data).

\begin{equation}\label{ch:rcnn:ex-donor}
  \begin{bmatrix}
    1 & 0 & 0 & 0 & 0 \\  % A
    0 & 0 & 0 & 1 & 0 \\  % G
    0 & 1 & 0 & 0 & 0 \\  % T
    0 & 0 & 0 & 0 & 1 \\ % N
    1 & 0 & 0 & 0 & 0 \\ % A
    1 & 0 & 0 & 0 & 0  % A
  \end{bmatrix}
\end{equation}

Matrix \ref{ch:rcnn:ex-donor} illustrates a six-nucleotide long,
one-hot-encoded DNA sequence window that reads AGTNAA, which contains the
consensus donor dinucleotide GT at position 1.

\section{\label{ch:rcnn:architecture}Neural Network Architecture}

This section reports selected splice site classification network architecture,
see Figure \ref{fig:rcnn:architecture} and explains the theoretical background
and motivation behind the input and each selected layer, as well as the
empirical findings.

\begin{figure}
  \centering
  \includegraphics[width=\textwidth,height=0.9\textheight]{figures/architecture.pdf}
  \caption{Model architecture visualization}
  \label{fig:rcnn:architecture}
\end{figure}

Table \ref{table:rcnn:win-size-donor} and Table
\ref{table:rcnn:win-size-acceptor} show the dependency between the shape of the
input window to the neural network and its performance. The tables clearly
demonstrate that the window overlap with the associated intron -- located at
the downstream direction for donors and upstream direction for acceptors --
plays a dominant role and that classifications performed on windows with no
overlap with the associated intron have very poor performance. This finding is
in agreement with the biological understanding of splicing, which is mostly
dependent on DNA sequences inside the intron. See Section
\ref{ch:background:dna-rna} which covers the biological background and Section
\ref{ch:evaluation:sensitivity} which explores the sensitivity of the trained
network at various input positions.

\begin{table}
  \begin{center}
    \begin{tabular}{ | l | d{2} | d{2} | d{2} | d{2} | d{2} | d{2} | }
      \hline

      &
      \multicolumn{2}{| l |}{\textbf{0}} &
      \multicolumn{2}{| l |}{\textbf{100}} &
      \multicolumn{2}{| l |}{\textbf{200}} \\

      \hline

      &
      \multicolumn{1}{| l |}{\textbf{Precision}} &
      \multicolumn{1}{| l |}{\textbf{Recall}} &
      \multicolumn{1}{| l |}{\textbf{Precision}} &
      \multicolumn{1}{| l |}{\textbf{Recall}} &
      \multicolumn{1}{| l |}{\textbf{Precision}} &
      \multicolumn{1}{| l |}{\textbf{Recall}} \\

      \hline
      \textbf{0} & \multicolumn{1}{| c |}{-} & \multicolumn{1}{| c |}{-} & 51.8\% & 85.2\% & 52.2\% & 85.4\% \\
      \hline
      \textbf{100} & 14.3\% & 32.2\% & 62.4\% & 87.8\% & 63.3\% & 88.0\% \\
      \hline
      \textbf{200} & 14.6\% & 34.0\% & 63.6\% & 88.1\% & 64.6\% & 87.9\% \\
      \hline
    \end{tabular}
  \end{center}
  \caption{\label{table:rcnn:win-size-donor}Dependency between classification
    performance on donors and the number of nucleotides upstream (rows) and
    downstream (columns) from the candidate splice site included in the input
    window.}
\end{table}

\begin{table}
  \begin{center}
    \begin{tabular}{ | l | d{2} | d{2} | d{2} | d{2} | d{2} | d{2} | }
      \hline

      &
      \multicolumn{2}{| l |}{\textbf{0}} &
      \multicolumn{2}{| l |}{\textbf{100}} &
      \multicolumn{2}{| l |}{\textbf{200}} \\

      \hline

      &
      \multicolumn{1}{| l |}{\textbf{Precision}} &
      \multicolumn{1}{| l |}{\textbf{Recall}} &
      \multicolumn{1}{| l |}{\textbf{Precision}} &
      \multicolumn{1}{| l |}{\textbf{Recall}} &
      \multicolumn{1}{| l |}{\textbf{Precision}} &
      \multicolumn{1}{| l |}{\textbf{Recall}} \\

      \hline
      \textbf{0} & \multicolumn{1}{| c |}{-} & \multicolumn{1}{| c |}{-} & 10.2\% & 22.1\% & 10.4\% & 24.3\% \\
      \hline
      \textbf{100} & 53.1\% & 86.3\% & 57.7\% & 87.2\% & 56.6\% & 87.2\% \\
      \hline
      \textbf{200} & 56.1\% & 86.4\% & 58.6\% & 87.3\% & 58.8\% & 87.4\% \\
      \hline
    \end{tabular}
  \end{center}
  \caption{\label{table:rcnn:win-size-acceptor}Dependency between
    classification performance on acceptors and the number of nucleotides
    upstream (rows) and downstream (columns) from the candidate splice site
    included in the input window.}
\end{table}

The convolutional layer allows for sparse connectivity between successive
layers and parameter sharing by applying the same kernel parameters to a
limited receptive field \cite[p.~330]{goodfellow2016deep}. This leads to
drastic reduction of the number of trainable parameters which, in turn,
decreases the computing time and needed training dataset size. Convolutions are
equivariant to translation \cite[p.~334]{goodfellow2016deep} and are therefore
theoretically capable of learning various features at different relative
positions inside input DNA sequence windows from fewer samples. It is
hypothesized that such position-independent features exist in the data, which
motivated the use of convolutional layers in the resulting network
architecture. It has been empirically verified that convolutional layers
improved various network performance metrics at low additional computational
costs.

Leaky rectified linear activation functions (Leaky ReLU) were used throughout
the network. Use of the ReLU activation function and its variations has many
benefits. ReLU was among the most frequently used and successful activation
function as of 2017 \cite{ramachandran2017searching}; it leads to extensive
research on them. Networks using ReLU units are more easily trained because
they do not have problems with vanishing and exploding gradients
\cite{ramachandran2017searching}. Leaky ReLU was used instead of plain ReLU
because it provides similar computational complexity but avoids 0 gradient
\cite{maas2013rectifier}, thereby leading to improved performance.

Skip connections forming residential networks (ResNet) are introduced in the
convolutional part of the network. Deep networks may suffer from the vanishing
gradient and the degradation problem, which could be resolved by using ResNet
\cite{he2016deep}.

Recurrent neural networks (RNNs) are capable of recognizing same features at
various positions in the input sequence and even recognize their repetition and
mutual dependence \cite[p.~367]{goodfellow2016deep}. To some extent, it is
possible to achieve similar position-independent feature recognition with
convolutional layers. However, their processing is shallower and disallows
greater interdependence between different positions in the sequence
\cite[p.~368]{goodfellow2016deep}. As of 2016, gated RNNs, which include
networks with long short-term memory (LSTM) units, were the most effective type
of RNNs \cite[p.~404]{goodfellow2016deep}. LSTM units allow the retention of
information over large number of time steps without any problems with vanishing
or exploding gradients \cite[p.~404]{goodfellow2016deep}. Other works also
report the successful use of RNN layers for splice site detection
\cite{lee2015dna}. For these reasons, a bidirectional RNN layer was used in the
selected architecture.

The work \cite{lee2015dna} reports no improvement after adding more recurrent
layers. The same finding of no statistically significant improvement after
adding another recurrent layer was observed in this work.

The used recurrent layer produces output at each step. This decision has been
motivated by the need to distinguish true splice sites that are located exactly
in the middle of the input sequence from those located in close proximity in
the middle. Bidirectional RNN was used to allow the recognition of both
downstream and upstream dependencies.

The aforementioned multilayered architecture allows for the learning of
complex, highly non-linear mappings between input DNA sequences and output
splice site confidences. The network has 49\,921 learnable parameters, which
give large learning capacity. However, neural networks with large capacities
tend to overfit the training data \cite{srivastava2014dropout}. This effect was
empirically observed and confirmed during experiments on neural network
architectures without sufficient regularization techniques. A dropout layer was
successfully used in the final architecture to completely overcome overfitting.

Dropout is a technique of omission of a given fraction of neural units in a
given layer during training. A different stochastic unit selection is done
during each training mini-batch. Classical optimization based on
backpropagation is then applied to this reduced network
\cite{srivastava2014dropout}. All neurons are applied with a scaling factor
during the later inference of a trained network. Dropout is an approximation of
an equally weighted geometric mean of the predictions of multiple neural
networks with shared parameters \cite{srivastava2014dropout}. The number of
dropouts and therefore the number of synthetic networks are exponential with
the number of neural units \cite{srivastava2014dropout}. This technique is
highly efficient from the point of view of computational resources and prevents
overfitting \cite{srivastava2014dropout}.

\section{\label{ch:rcnn:criteria}Model Criteria}

Model selection was based on the following criteria:

\begin{itemize}
  \item ability of the model to generalize and perform consistently among
    different fungal organisms,
  \item good performance metrics, namely its precision and recall,
  \item model complexity and computational intensity during inference.
\end{itemize}

The generalization capability of the model was an important aspect because the
model would be applied to a wide variety of previously unknown fungal genomes.
For this reason, all experimental models were evaluated on individual organisms
separately as well as on larger sets of organisms. Consistency was taken into
account in this regard.

Precision and recall were used to evaluate the performance of the models. Small
false positive rate was emphasized during the development of the network
architecture and the selection of binary classification threshold values. This
is because consensus dinucleotides are more than an order of magnitude more
frequent in target DNA sequences than true splice sites, see Section
\ref{ch:data:stats} for more details.

Final production computational resource utilization was also an important
consideration during the design of the network architecture and pre-processing
and post-processing procedures. This aspect is important because metagenomes on
which they will be applied are enormous and might go well beyond $10^9$
nucleotides. The possible use of the results of this work in an automated
online annotation tool is another consideration, and high computational
requirement would make such a goal economically unfeasible. This criterion has
led to the selection of a sub-optimal network from the point of view of
accuracy as networks with larger complexity and window size had slightly better
results. But the improvement was not significant and would lead to a large
increase in computational requirements.

\section{\label{ch:rcnn:splice-site-training}Neural Network Training}

Stochastic gradient descent (SGD) was used for the optimization during network
training. Other optimization techniques, such as the adaptive methods AdaGrad,
RMSProp, or Adam, were also tested, but SGD yielded superior results. It has
been reported that adaptive methods converge faster during the initial phases
of training; but given enough training time, it may lead to worse
generalization \cite{wilson2017marginal}. This effect was empirically confirmed
in this work.

The initial learning rate was set to $\alpha = 0.01$ and multiplied by the
factor of $0.2$ after every epoch which did not lead to a decrease in
validation loss. The training was automatically stopped after 10 successive
epochs with no significant decrease in validation loss. The model with the
lowest after-epoch validation loss was used during successive evaluation and
tests.

Binary cross-entropy was chosen as a surrogate loss function.

It has been reported that small batch sizes ($m \le 32$) lead to better test
results and generalization. For some networks and tasks, this effect might go
down to the batch size as small as $m = 2$ \cite{masters2018revisitingb}. The
batch size of $m = 16$ was used during training.

\begin{figure}
  \centering
  \includegraphics[width=0.8\textwidth]{figures/donor_evaluation/training-loss.pdf}
  \caption{Training and validation loss during training of the donor model}
  \label{fig:rcnn:training-loss}
\end{figure}

Over three million samples from 769 organisms were used in the training dataset
and over 85\,000 samples from 85 organisms were used in the validation dataset.
See Section \ref{ch:rcnn:dataset-size} that talks about validation dataset size
selection. After each epoch, training samples were randomly reshuffled.
Training and validation loss progression is illustrated by Figure
\ref{fig:rcnn:training-loss}.

Keras \cite{chollet2015keras} with TensorFlow \cite{abadi2016tensorflow}
backend was used for training and prediction. Preprocessed data in the form of
Numpy NPZ files with input matrices and desired outputs is used during
training, see Section \ref{ch:automation:samples}. Data was loaded gradually
from the disk and not kept in memory because of the large number of samples
used during the training.

See Chapter \ref{ch:evaluation} for detailed evaluation of achieved results.

\section{\label{ch:rcnn:dataset-size}Dataset Size}

The validation dataset always consisted of 50\% of negative samples and 50\% of
positive samples. The validation dataset size $l$ was chosen so that the
probability of empirical true positive rate (TPR) or empirical true negative
rate (TNR) being more than $\epsilon$, different from the true TPR or the true
TNR in any of $n$ experiments, was smaller or equal to $R$. See Section
\ref{ch:background:evaluation} for more information on evaluation.

Formula \ref{eq:rcnn:proba-single} gives the maximum probability $R_1$ of
seeing a “bad” TPR or TNR measurement in a single evaluation for the above
condition to hold. This is derived from the fact that $2 \cdot n$ measurements
need to be performed to obtain empirical TPR and empirical TNR for $n$
different model setups.

Formula \ref{eq:rcnn:hoeffding} is derived from Hoeffding's inequality
\cite{hoeffding1994probability} and gives the minimum dataset size $l$. The
right part of the equation is multiplied by 2 because the dataset is split into
half between positive and negative samples.

A dataset comprising $l = 75759$ samples is needed for $n = 25$, $R = 0.05$
(5\%) and $\epsilon = 0.01$ (1\%).

\begin{equation}
  R_1 = 1 - (1 - R)^\frac{1}{2 \cdot n}
  \label{eq:rcnn:proba-single}
\end{equation}

\begin{equation}
  l = \ceil*{2 \cdot \frac{\log 2 - \log R_1}{2 \cdot \epsilon^2}}
  \label{eq:rcnn:hoeffding}
\end{equation}

\chapter{\label{ch:automation}Automation}

\minitoc

This chapter describes various automation software developed as part of this
work, including data pre-processing and extraction, neural network training,
and evaluation. It explains how the work was split into a multi-step pipeline
and further describes some technical details that have implications on how the
neural networks were trained or evaluated.

Source data, as described in Chapter \ref{ch:data}, contains assembled DNA
sequence scaffolds in FASTA files and various feature annotations in general
feature format (GFF) files. A multistep data pre-processing pipeline was
created to extract target training, evaluation, and testing data from the
source data. This pipeline is split into multiple steps to reduce its
complexity, increase the time-efficiency, and to enable efficient
reproducibility with varying configuration parameters like extracted window
size. A set of fully automated training, evaluation, inspection, and
visualization scripts were also created.

Throughout the work, 0-based indexing is used for sequence positions and start
inclusive, end exclusive intervals are used.

All source codes are published in GitHub repository
\url{https://github.com/Indy2222/introns/}. Source data, trained neural
networks, and other large binary materials are uploaded to Google Cloud Storage
which is linked from the repository.

\section{\label{ch:automation:overview}Overview}

In the text of this chapter, in the source code and in file naming the term
``gene'' is often used for a continuous area withing a DNA sequence that spans
from the beginning of the first annotated CDS to the end of the last annotated
CDS of an actual gene. This is a simplification that does not fully correspond
to actual genes in all their complexities as understood by biology.

The pre-processing pipeline consists of the following steps:

\begin{enumerate}
  \item Within each organism, start and end positions of individual genes and
    introns are extracted. These are collectively referred to as feature, see
    Section \ref{ch:automation:features}.
  \item Positions of positive and negative examples of donor and acceptor
    Splice sites are generated. See Section \ref{ch:automation:positions}.
  \item Final data samples, which map DNA sequence windows to scores $0.0$ or
    $1.0$, are created. See Section \ref{ch:automation:samples}.
\end{enumerate}

The data pre-processing pipeline is implemented in the Rust programming
language for its high performance, reliability, and ``fearless concurrency''
\cite{matsakis2014rust}\cite{rust-web}. All CPU-intensive parts of the pipeline
are parallelized to improve the processing time on multicore computers.

Training, evaluation, inspection, and visualization are described in Section
\ref{ch:automation:evaluation}.

Diagram of the preprocessing, training and evaluation pipeline is depicted in
Figure \ref{fig:automation:pipeline}.

\begin{figure}
  \centering
  \includegraphics[width=\textwidth,height=0.9\textheight]{figures/pipeline.pdf}
  \caption{Diagram of data extraction and preprocessing, splice site
    classification model training and evaluation.}
  \label{fig:automation:pipeline}
\end{figure}

\section{\label{ch:automation:features}Intron and Gene Extraction}

Start and end positions of individual protein coding genes and introns are
extracted from the source GFF files by iterating over a sorted list of all
annotated coding sequences (CDS). Each CDS has a protein ID attribute, a start
position, an end position, and a scaffold name. Gene boundaries are obtained
from the start position of its first CDS and the end position of its last CDS.
All gaps within the gene not covered by any of its CDS are considered and
stored as intros. See Figure \ref{fig:automation:color-coded} illustrating
extracted and non-extracted introns.

\begin{figure}
\centering
\begin{BVerbatim}[commandchars=\\\{\}]
AGT\textcolor{cyan}{ATCATGA}\textcolor{magenta}{AGGAA}\textcolor{cyan}{GAACA}\textcolor{blue}{AGTTGA}\textcolor{red}{GTGACATAATTACCAG}\textcolor{blue}{GGGT}\textcolor{cyan}{GCG}GCG
\end{BVerbatim}
\caption{An illustration of a DNA sequence with UTR of exons highlighted in
  \textcolor{cyan}{cyan}, CDS highlighted in \textcolor{blue}{blue}, intra-UTR
  introns (not extracted) highlighted in \textcolor{magenta}{magenta} and
  intra-CDS introns (extracted) highlighted in \textcolor{red}{red}.}
\label{fig:automation:color-coded}
\end{figure}

CDS makes a subset of exons but not vice versa, as exons close to both the
5\textquotesingle{} and 3\textquotesingle{} sites of a gene contain UTR and may
contain only UTR \cite{bicknell2012introns}. The use of CDS for intron
detection implies that only intra-CDS introns are extracted.

Only genes and introns on the positive strand were considered in the data
extraction pipeline and data analysis. This simplifies the pipeline but reduces
the amount of extracted data into half. The smaller dataset without reduction
of variability in the data does not have any impact on the results due to the
large size of the dataset. See Section \ref{ch:data:stats} with dataset
statistics.

Gene direction is close to random in fungal DNA \cite{li2012gene}, and it is
further supposed that intron-specific differences between two DNA strands are
weak or non-existent. This possible effect on the data is further weakened by
the non-systematic selection of strands during DNA sequencing. However, some of
these assumptions deserve a deeper analysis.

A CSV file with genes and introns is produced for each organism. Each gene and
intron has an ID and a parent ID---this allow detailed analysis at later
stages.

\section{\label{ch:automation:positions}Candidate Splice Site Extraction}

Positions of all candidate splice sites within genes ,as described by Section
\ref{ch:automation:features}, are detected. Candidate splice sites are divided
into donors and acceptors; they are further divided into positive
examples---i.e. where true splice sites are located---and negative examples.
Not all candidate splice sites are included, see Section
\ref{ch:rcnn:candidates}. The intersecting areas of overlapping genes are
processed only once to avoid the duplication of candidate splice site
positions. Splice sites from all genes are considered within these areas.

Consensus dinucleotide, which does coincide with a respective splice site of an
intron of any gene, is not included in the set of negative splice site samples
even if it is included in another overlapping gene at a non-splice site
position. This prevents inconsistent labeling of training data in cases of
alternative splicing and overlapping genes.

Four Numpy NPZ files are generated for each organism:

\begin{itemize}
  \item positive donor positions,
  \item positive acceptor positions,
  \item negative donor positions,
  \item negative acceptor positions.
\end{itemize}

Each of these NPZ files maps the scaffold name to a list of 0-indexed positions
and feature IDs.

\section{\label{ch:automation:samples}Training Samples Extraction}

Candidate splice site positions, as described in Section
\ref{ch:automation:positions}, are used for the extraction of actual training,
validation, and testing data. The generation of samples is the first step in
the pre-processing pipeline which includes some kind of sub-sampling (i.e. no
positions are skipped by the previous pipeline steps). The intermediate data is
utilized to avoid the scanning of large amounts of source files with DNA
sequences and annotations when regenerating training data on various filtering,
sampling, and windowing criteria.

Sample window size, i.e. the number of nucleotides in it, splice site relative
offset, the maximum number of examples, and other criteria are all
configurable. Stratified sampling is done on groups defined by organism, splice
site type, and true/false positivity to avoid any over-representation of
organisms. Under this constraint, examples were selected stochastically.

The samples are stored in numbered sub-sub directories as NPZ files map the
encoded input sequences to labels. This nested structure is used to obtain good
performance on some Linux file systems as the performance on large directories
might be very low \cite{djordjevic2012ext4}. A CSV index file, coupling the
file paths with sample types and other properties, is created for quick access
and further sub-sampling for experiments that do not require a large amount of
data.

\section{\label{ch:automation:evaluation}Training and Evaluation}

There is a script for automatic neural network training, other script for
generation and persistence of training-validation-test data split with various
other dataset configurations.

A program which automatically evaluates a trained neural network on validation
and test datasets was created to allow fast experimentation and comparability
among experiments. A PDF with various network performance metrics on whole
datasets as well as on individual organisms is produced by the program. This
program is extended with more programs that do more detailed inspections,
evaluations, and visualization---for example, the analysis of sensitivity of
the networks to one nucleotide mutations.

Training, evaluation, and other related automation are implemented in Python
for its ease of use, widespread usage among scientists, and abundance of
relevant deep learning, statistical and data science libraries.

\chapter{\label{ch:evaluation}Evaluation}

\minitoc

Detailed evaluation, inspection, and analysis of splice site classification
networks, as well as the performance of the overall intron detection pipeline,
is reported and discussed in this chapter.

Section \ref{ch:evaluation:datasets} gives details of test sample selection and
puts that into the context of other existing research, specific needs, and
goals of this work. Section \ref{ch:evaluation:basic} reports several basic
per-organism evaluation metrics, such as precision and recall, as well as
prediction confidence distributions. Section \ref{ch:evaluation:comparison}
compares the performance of the neural networks with the SVM developed in
\cite{barucic}. Section \ref{ch:evaluation:cpu} describes the CPU usage of the
neural networks and SVM; it also evaluates the possible cost of using the
pipeline in the Google Cloud Platform.

Section \ref{ch:evaluation:sensitivity} analyzes the sensitivity of the neural
networks to single nucleotide mutations/swaps and compares that to known
biological phenomena. Section \ref{ch:evaluation:proximity} reports the error
rate of the networks on negative samples in proximity to a true splice site.
Section \ref{ch:evaluation:lengths} describes the error rate of the network on
introns of various lengths. Section \ref{ch:evaluation:correlation} evaluates
the dependency of donor and acceptor splice site models. And finally, Section
\ref{ch:evaluation:whole} discusses the overall gene prediction pipeline.

\section{\label{ch:evaluation:datasets}Datasets}

Prior to all experiments, the data was split into training, validation, and
test datasets. The performance of splice site classification networks was
evaluated on organisms that were not included in the training and validation
datasets. The test dataset consists of eight organisms, each belonging to a
different phylum. Data from organisms included in training datasets was not
used in any experiment or evaluation before the final evaluation reported in
this chapter. The selected test organisms are the same as in the work
\cite{barucic} to allow for good comparison. See Table
\ref{table:evaluation:test-set} which lists all organisms included in the test
dataset.

\begin{table}
  \begin{center}
    \begin{tabular}{ | l | l | l | }
      \hline
      \textbf{Phylum} & \textbf{Species} & \textbf{Organism ID} \\
      \hline
      Ascomycota & Aspergillus wentii & Aspwe1 \\
      Basidiomycota & Mycena albidolilacea & Mycalb1  \\
      Blastocladiomycota & Allomyces macrogynus & Allma1 \\
      Chytridiomycota & Chytriomyces sp. MP71 & Chytri1 \\
      Cryptomycota & Rozella allomycis & Rozal\_SC1 \\
      Microsporidia & Encephalitozoon hellem & Enche1 \\
      Mucoromycota & Lichtheimia corymbifera & Liccor1 \\
      Zoopagomycota & Coemansia reversa & Coere1 \\
      \hline
    \end{tabular}
  \end{center}
  \caption{\label{table:evaluation:test-set}Organisms included in the test dataset}
\end{table}

The decision to split datasets on the organism lever rather than the sample
level is motivated by the need to test the network ability in order to
generalize previously unseen organisms. The network will be used in a pipeline
executed on metagenomes containing large number of previously unseen organisms.
This sharply contrasts with some studies on automated intron detection because
it usually reports the results measured on organisms included in the training
dataset \cite{zuallaert2018splicerover}.

\section{\label{ch:evaluation:basic}Basic Metrics}

One final neural network architecture was selected for the classification of
both candidate donor splice sites and candidate acceptor splice sites. See
Section \ref{ch:rcnn:architecture}. Two versions of the model with different
input window sizes were selected for the final evaluation and usage. Model
NN\,100 has an input window size of 100 nucleotides and model NN\,400 has an
input window size of 400 nucleotides. The larger model, unsurprisingly, has
better performance but at the cost of roughly $2.9\times$ higher CPU usage per
classification. See Table \ref{table:rcnn:win-size-donor} and Table
\ref{table:rcnn:win-size-acceptor} which compare the performance of the models
with various input window sizes (evaluated on the validation dataset during the
experimentation phase).

The performances of donor splice site classification models are reported in
Table \ref{table:evaluation:donor} and the performances of acceptor splice site
models are reported in Table \ref{table:evaluation:acceptor}
\tablefootnote{Only 20 positive donor samples and 19 positive acceptor samples
  were used during the evaluation of organism Enche1.}. The low precision on
some organisms is largely due to the high ratio of the number of candidate
splice sites to the number of true splice sites. See Section
\ref{ch:data:stats} on data statistics. Also see Section
\ref{ch:rcnn:candidates} that describes which splice sites were included in the
datasets.

\begin{table}
  \begin{center}
    \begin{tabular}{ | l | d{2} | d{2} | d{2} | d{2} | d{2} | d{2} | }
      \hline
      & \multicolumn{3}{| l |}{\textbf{NN\,100}} & \multicolumn{3}{| l |}{\textbf{NN\,400}} \\
      \hline

      & \multicolumn{1}{| l |}{\textbf{Precision}}
      & \multicolumn{1}{| l |}{\textbf{Recall}}
      & \multicolumn{1}{| l |}{\textbf{AUC}}
      & \multicolumn{1}{| l |}{\textbf{Precision}}
      & \multicolumn{1}{| l |}{\textbf{Recall}}
      & \multicolumn{1}{| l |}{\textbf{AUC}} \\

      \hline
      Allma1      & 55.8\% & 74.0\% & 96.1\% & 62.8\% & 82.3\% & 97.2\% \\
      Aspwe1      & 53.3\% & 92.5\% & 98.3\% & 68.1\% & 94.0\% & 98.5\% \\
      Chytri1     & 49.4\% & 71.4\% & 95.5\% & 65.1\% & 85.0\% & 97.7\% \\
      Coere1      &  7.1\% & 61.8\% & 85.5\% &  9.0\% & 64.7\% & 87.8\% \\
      Enche1      &  0.8\% & 100.0\% & 99.4\% & 1.0\% & 95.0\% & 98.2\% \\
      Liccor1     & 56.2\% & 89.9\% & 98.0\% & 75.8\% & 94.8\% & 98.9\% \\
      Mycalb1     & 61.9\% & 78.1\% & 95.2\% & 75.3\% & 82.4\% & 96.6\% \\
      Rozal\_SC1  & 25.6\% & 40.3\% & 86.6\% & 35.7\% & 45.3\% & 90.0\% \\
      \hline
    \end{tabular}
  \end{center}
  \caption{\label{table:evaluation:donor}Performance of donor classification
    models. Model NN\,100 has input window size of 100 nucleotides, starting at
    the splice site and going downstream. Model NN\,400 has input window size
    of 400 nucleotides, spanning exactly 200 nucleotides to both sides from the
    splice site.}
\end{table}

\begin{table}
  \begin{center}
    \begin{tabular}{ | l | d{2} | d{2} | d{2} | d{2} | d{2} | d{2} | }
      \hline
      & \multicolumn{3}{| l |}{\textbf{NN\,100}} & \multicolumn{3}{| l |}{\textbf{NN\,400}} \\
      \hline

      & \multicolumn{1}{| l |}{\textbf{Precision}}
      & \multicolumn{1}{| l |}{\textbf{Recall}}
      & \multicolumn{1}{| l |}{\textbf{AUC}}
      & \multicolumn{1}{| l |}{\textbf{Precision}}
      & \multicolumn{1}{| l |}{\textbf{Recall}}
      & \multicolumn{1}{| l |}{\textbf{AUC}} \\

      \hline
      Allma1      & 70.6\% & 80.2\% & 97.2\% & 66.3\% & 83.7\% & 97.4\% \\
      Aspwe1      & 54.6\% & 93.4\% & 98.5\% & 56.9\% & 93.9\% & 98.6\% \\
      Chytri1     & 59.1\% & 82.0\% & 97.3\% & 60.3\% & 84.9\% & 97.7\% \\
      Coere1      &  5.9\% & 46.6\% & 83.5\% &  7.1\% & 57.0\% & 85.3\% \\
      Enche1      &  0.3\% & 42.1\% & 91.9\% &  0.3\% & 42.1\% & 94.3\% \\
      Liccor1     & 68.5\% & 93.5\% & 98.8\% & 72.3\% & 93.8\% & 98.9\% \\
      Mycalb1     & 65.4\% & 80.8\% & 96.3\% & 63.9\% & 83.0\% & 96.7\% \\
      Rozal\_SC1  & 12.0\% & 22.7\% & 79.6\% & 19.5\% & 30.1\% & 84.1\% \\
      \hline
    \end{tabular}
  \end{center}
  \caption{\label{table:evaluation:acceptor}Performance of acceptor
    classification models. Model NN\,100 has input window size of 100
    nucleotides, starting 100 nucleotides upstream from the splice site and
    going downstream. Model NN\,400 has input window size of 400 nucleotides,
    spanning exactly 200 nucleotides to both sides from the splice site.}
\end{table}

The neural networks output values between 0 and 1. This value needs to be
compared to a threshold to obtain a binary classification. During the
evaluation of all splice site classification models, the threshold was set to
$0.5$. See Figure \ref{fig:evaluation:donor-cdf} and Figure
\ref{fig:evaluation:donor-density} which depict the distribution of the
prediction values on both positive and negative samples of the donor model.
Figure \ref{fig:evaluation:acceptor-cdf} and Figure
\ref{fig:evaluation:acceptor-density} visualize the same properties on the
acceptor model.

The confidence intervals visualized in Figure
\ref{fig:evaluation:donor-density} and Figure
\ref{fig:evaluation:acceptor-density} are calculated with Formula
\ref{eq:evaluation:ci-low} and Formula \ref{eq:evaluation:ci-high} derived in
\cite[p.~176]{johnson2005univariate}.

\begin{equation}
  \theta_L = 0.5 \chi^2_{2k, \alpha / 2}
  \label{eq:evaluation:ci-low}
\end{equation}

\begin{equation}
  \theta_U = 0.5 \chi^2_{2k + 2, 1 - \alpha / 2}
  \label{eq:evaluation:ci-high}
\end{equation}

Kolmogorov–Smirnov statistic computed on the cumulative distribution functions
of the prediction values on positive and negative samples is $0.83$ at $0.18$
for the donor model and $0.79$ at $0.18$ for the acceptor model. The values
imply the high ability of the model to discriminate positive and negative
samples at the threshold point of $0.18$. During the evaluation and production
use of the models, a higher threshold of $0.5$ was used to compensate for the
high positive to negative sample rate ratio.

The prediction distributions were measured over samples from all organisms
included in the test dataset. In total, 10\,000 positive samples and 10\,000
negative samples were included from each organism, except those which did not
have enough annotated features. The only organism which was largely
underrepresented was Enche1. Prediction distribution on different organisms,
especially organisms from different phyla, differ. Likewise, the used sample
set does not represent true organism distribution in nature or in extracted
metagenomes. This implies the need for a careful interpretation of the plots.

\begin{figure}
  \centering
  \includegraphics[width=0.8\textwidth]{figures/donor_evaluation/cdf.pdf}
  \caption{Cumulative distribution function with Kolmogorov–Smirnov statistic
    of prediction values from NN\,400 model on positive and negative samples of
    donor splice sites. }
  \label{fig:evaluation:donor-cdf}
\end{figure}

\begin{figure}
  \centering
  \includegraphics[width=0.8\textwidth]{figures/donor_evaluation/density.pdf}
  \caption{Density of prediction values from NN\,400 model on positive and
    negative samples of donor splice sites.}
  \label{fig:evaluation:donor-density}
\end{figure}

\begin{figure}
  \centering
  \includegraphics[width=0.8\textwidth]{figures/acceptor_evaluation/cdf.pdf}
  \caption{Cumulative distribution function with Kolmogorov–Smirnov statistic
    of prediction values from NN\,400 model on positive and negative samples of
    acceptor splice sites.}
  \label{fig:evaluation:acceptor-cdf}
\end{figure}

\begin{figure}
  \centering
  \includegraphics[width=0.8\textwidth]{figures/acceptor_evaluation/density.pdf}
  \caption{Density of prediction values from NN\,400 model on positive and
    negative samples of acceptor splice sites.}
  \label{fig:evaluation:acceptor-density}
\end{figure}

\section{\label{ch:evaluation:comparison}Comparison with SVM}

Table \ref{table:evaluation:comp-svm} compares the true positive rate and true
negative rate of two neural networks with different input window sizes and SVM,
as initially developed in \cite{barucic}. The smaller neural network (NN\,100)
had an input window of 100 nucleotides going from the splice site in the intron
direction (downstream in the case of a donor and upstream in the case of an
acceptor). The larger neural network (NN\,400) had a window size of 400
nucleotides and spanned 200 nucleotides to both directions from the splice
site. The SVM used for the evaluation was trained solely on data from the
Basidiomycota phylum.

Both neural networks produce much less false positive predictions compared to
the SVM. Furthermore, the NN\,400 model outperforms the SVM in every
measurement, except for $\mathit{TPR}$ on donors in the Basidiomycota phylum.

The improvement in $\mathit{TNR}$ achieved with the neural networks is very
important due to the large ratio of the false candidate splice sites to the
true candidate splice sites.

All classification methods in Table \ref{table:evaluation:comp-svm} were
evaluated on the same dataset. Negative samples were taken from genetic regions
defined as the area between the beginning of the first exon of the gene and the
end of the last exon of the gene. These regions are wider than the CDS and
intra-CDS regions used in other parts of this work, see Section
\ref{ch:automation:overview}.

\begin{table}
  \begin{center}
    \begin{tabular}{ | l | l | d{2} | d{2} | d{2} | d{2} | d{2} | d{2} | }
      \hline

      & &
      \multicolumn{3}{| l |}{\textbf{Ascomycota}} &
      \multicolumn{3}{| l |}{\textbf{Basidiomycota}} \\

      \hline

      & &
      \multicolumn{1}{| l |}{\textbf{SVM}} &
      \multicolumn{1}{| l |}{\textbf{NN\,100}} &
      \multicolumn{1}{| l |}{\textbf{NN\,400}} &
      \multicolumn{1}{| l |}{\textbf{SVM}} &
      \multicolumn{1}{| l |}{\textbf{NN\,100}} &
      \multicolumn{1}{| l |}{\textbf{NN\,400}} \\

      \hline

      \multirow{2}{*}{\textbf{Donor}}
      & $\mathit{TPR}$ & 86.6\% & 84.9\% & 87.2\% & 91.6\% & 86.3\% & 89.9\% \\
      & $\mathit{TNR}$ & 94.9\% & 97.2\% & 97.7\% & 95.9\% & 97.6\% & 98.1\% \\

      \hline

      \multirow{2}{*}{\textbf{Acceptor}}
      & $\mathit{TPR}$ & 83.4\% & 85.3\% & 86.4\% & 88.1\% & 87.7\% & 88.8\% \\
      & $\mathit{TNR}$ & 93.8\% & 97.7\% & 97.9\% & 93.8\% & 97.8\% & 97.7\% \\

      \hline
    \end{tabular}
  \end{center}
  \caption{\label{table:evaluation:comp-svm}Comparison of true positive rate
    ($\mathit{TPR}$) and true negative rate ($\mathit{TNR}$) between SVM,
    neural network with window size 100 (NN\,100) and neural network with
    window size 400 (NN\,400).}
\end{table}

\section{\label{ch:evaluation:cpu}Computational Intensity}

The NN\,400 model uses $2.88\times$ more CPU time per inference than the
NN\,100 model. The SVM uses $46.4\times$ more CPU time per inference than the
NN\,100 model. This makes the use of the neural networks more practical on
large metagenomes.

The NN\,100 model uses 1.37 CPU seconds per 1000 predictions, while the NN\,400
model uses 3.94 CPU seconds per 1000 predictions. The measurement was performed
on the laptop ThinkPad T490 with Intel\textsuperscript{\tiny\textregistered}
Core™ i5-8265U CPU @ 1.60GHz. The measurement was performed with all samples
loaded in memory. TensorFlow version 2.3.1 compiled with instructions
\Verb_AVX2_ and \Verb_FMA_ enabled was used. The measurement included only the
CPU time usage of Python call \Verb_model.predict(inputs)_.

There is around $10^8$ donor and acceptor candidate splice sites in a
metagenome of size of $10^9$ nucleotides. Using the NN\,100 model, all
candidate splice sites in this hypothetical dataset could be classified in
137\,000 CPU seconds or 9.5 hours on four CPU cores. Both CPU and real time
would be much smaller on a server CPU, GPU or TPU.

One hour of a single CPU on an E2 machine type in Google Cloud Platform costs
0.021811\,USD \cite{gcp-pricing}. Using this price and ignoring the differences
between CPUs, any inefficiencies, or overhead, all introns in the
aforementioned hypothetical metagenome could be classified for 0.83\,USD.

\section{\label{ch:evaluation:sensitivity}Positional Sensitivity}

Figure \ref{fig:evaluation:donor-divergence-all} visualizes an estimation of
the Kullback–Leibler divergence between the distributions of donor model
inferences made on unmodified sequences and sequences with single nucleotide
symbol swaps.

\begin{figure}
  \centering
  \includegraphics[width=0.8\textwidth]{figures/donor_evaluation/divergence-all.pdf}
  \caption{Kullback–Leibler divergences between NN\,400 donor model inferences
    on unmodified input sequences and inferences on sequences with single
    nucleotide modifications at various positions. Position-wise maximum over
    swaps to adenine, thymine, cytosine, adenine are shown.}
  \label{fig:evaluation:donor-divergence-all}
\end{figure}

The figure illustrates that the network is highly sensitive in the close
vicinity of the splice site with decreasing sensitivity in the downstream
(intron) direction. Almost no sensitivity could be found to one nucleotide
swaps 9 nucleotides and further upstream from the splice site. Part of this
sensitivity spike around the splice site is consistent with the literature
which reports that the 5\textquotesingle{}-terminus of the U1 snRNA component
of spliceosome binds to the nearly perfect Watson-Crick complement sequence
\Verb_CAGGURAGU_ that spans -3 to +6 around 5\textquotesingle{}-end of an
intron \cite{de2013exon}.

The second peak of sensitivity is found around 40 nucleotides downstream from
the splice site. There is a high frequency of introns of a length between 40
and 75 nucleotides, see Figure \ref{fig:data:intron-len-dist}. The branch point
is located 18 to 40 nucleotides upstream from the acceptor splice site
\cite{clancy2008rna}. The relative location of the second peak likely implies
that the network is learned to recognize area around branch point. This is
further supported by Figure \ref{fig:evaluation:donor-divergence-60}, where
positive samples were limited to the splice sites of introns of a length of 60
nucleotides. Figure \ref{fig:evaluation:donor-divergence-60} also displays a
sensitivity peak at the location of the acceptor splice site.

\begin{figure}
  \centering
  \includegraphics[width=0.8\textwidth]{figures/donor_evaluation/divergence-60.pdf}
  \caption{Kullback–Leibler divergences between NN\,400 donor model inferences
    on unmodified input sequences and inferences on sequences with single
    nucleotide modifications at various positions. Position-wise maximum over
    swaps to adenine, thymine, cytosine, adenine are shown. Positive samples
    are limited to the splice site of introns of a length 60 nucleotides.}
  \label{fig:evaluation:donor-divergence-60}
\end{figure}

Figure \ref{fig:evaluation:acceptor-divergence-all} visualizes an estimation of
the Kullback–Leibler divergence between the distributions of acceptor model
inferences made on unmodified sequences and sequences with single nucleotide
symbol swaps. Figure \ref{fig:evaluation:acceptor-divergence-60} visualizes the
divergence on introns of a length of 60 nucleotides. The acceptor sensitivity
plot largely overlaps with the donor sensitivity plot, with a notable
difference of the sensitivity decreasing more steeply near the opposite splice
site and vice versa. Compared to the donor model, the acceptor model is more
sensitive to one nucleotide swap in the negative samples.

\begin{figure}
  \centering
  \includegraphics[width=0.8\textwidth]{figures/acceptor_evaluation/divergence-all.pdf}
  \caption{Kullback–Leibler divergences between NN\,400 acceptor model
    inferences on unmodified input sequences and inferences on sequences with
    single nucleotide modifications at various positions. Position-wise maximum
    over swaps to adenine, thymine, cytosine, adenine are shown.}
  \label{fig:evaluation:acceptor-divergence-all}
\end{figure}

\begin{figure}
  \centering
  \includegraphics[width=0.8\textwidth]{figures/acceptor_evaluation/divergence-60.pdf}
  \caption{Kullback–Leibler divergences between NN\,400 acceptor model
    inferences on unmodified input sequences and inferences on sequences with
    single nucleotide modifications at various positions. Position-wise maximum
    over swaps to adenine, thymine, cytosine, adenine are shown. Positive
    samples are limited to the splice site of introns of a length of 60
    nucleotides.}
  \label{fig:evaluation:acceptor-divergence-60}
\end{figure}

A similar analysis based on different techniques was done in
\cite{zuallaert2018splicerover} which used a CNN trained and evaluated on
Arabidopsis and human. Visualizations reported in that paper, however, differ
with the results reported in this section, especially in the areas more than 10
nucleotides distant from the splice sites.

Kullback–Leibler divergence estimation is calculated with Formula
\ref{ch:evaluation:divergence} on two $n$-tuples of samples i.i.d. drawn from
distributions $p$ and $q$ respectively. $\nu_k(i)$ is distance of the $i$-th
sample from the first $n$-tuple to the $k$-th nearest neighbor from the second
$n$-tuple; $\rho_k(i)$ is distance of the $i$-th sample from the first
$n$-tuple to the $k + 1$ nearest neighbor from the same $n$-tuple. This
equation was derived from \cite{wang2006nearest}, $n = 1000$ and $k = 10$ were
used.

\begin{equation}
  D_{n}(p \parallel q) = \frac{1}{n} \sum_{i = 1}^n log
  \frac{\nu_k(i)}{\rho_k(i)} + \log \frac{n}{n - 1}
  \label{ch:evaluation:divergence}
\end{equation}

\section{\label{ch:evaluation:proximity}False Positives in the Proximity to a Splice Site}

Figure \ref{fig:evaluation:donor-splice-site-dist-wide} shows the dependence of
the donor model performance on negative splice site examples to the distance to
the closest real donor splice site. Figure
\ref{fig:evaluation:donor-splice-site-dist-narrow} displays this dependence
only in a narrow neighborhood of true splice sites.

The plots display the mean predicted value and the false positive rate for each
distance bin. The mean predicted value is equal to the mean prediction error
because only negative samples are used. The measurements are done on bins of
size 5 in Figure \ref{fig:evaluation:donor-splice-site-dist-wide} and on bins
of size 3 in Figure \ref{fig:evaluation:donor-splice-site-dist-narrow}. The
figures differ due to the overall captured distance range and different bin
sizes.

This data was generated on the validation dataset because there were not enough
samples in the test dataset to calculate the unbiased report without too much
noise. Only splice sites with consensus dinucleotides were included among the
negative samples. However, the closest true splice site was selected from the
set of all splice sites.

\begin{figure}
  \centering
  \includegraphics[width=0.8\textwidth]{figures/donor_evaluation/splice-site-dist-wide.pdf}
  \caption{NN\,400 donor model error rate dependence on the relative position
    of the nearest true donor splice site.}
  \label{fig:evaluation:donor-splice-site-dist-wide}
\end{figure}

\begin{figure}
  \centering
  \includegraphics[width=0.8\textwidth]{figures/donor_evaluation/splice-site-dist-narrow.pdf}
  \caption{NN\,400 donor model error rate dependence on relative position of
    the nearest true donor splice site.}
  \label{fig:evaluation:donor-splice-site-dist-narrow}
\end{figure}

Figure \ref{fig:evaluation:acceptor-splice-site-dist-wide} and Figure
\ref{fig:evaluation:acceptor-splice-site-dist-narrow} visualize the same
dependency for the acceptor splice site model. An interesting difference from
the donor model is that the error spike around the true splice site is higher
and wider.

\begin{figure}
  \centering
  \includegraphics[width=0.8\textwidth]{figures/acceptor_evaluation/splice-site-dist-wide.pdf}
  \caption{NN\,400 acceptor model error rate dependence on the relative
    position of the nearest true donor splice site.}
  \label{fig:evaluation:acceptor-splice-site-dist-wide}
\end{figure}

\begin{figure}
  \centering
  \includegraphics[width=0.8\textwidth]{figures/acceptor_evaluation/splice-site-dist-narrow.pdf}
  \caption{NN\,400 acceptor model error rate dependence on the relative
    position of the nearest true donor splice site.}
  \label{fig:evaluation:acceptor-splice-site-dist-narrow}
\end{figure}

The error spike around true splice sites leads to the higher than uniform
presence of false intron detections with an almost perfect overlap with the
true introns compared to introns with a low overlap with the true introns.

After the detected splice sites are combined into whole introns, among the
overlapping intron detections only those detected with the largest splice site
prediction values are kept. It is expected that this would lead to the
selection of the true introns in the majority of the cases.

Likewise, a small negative effect on gene prediction from incorrect intron
detections which large relative overlap with the true introns is expected.

\section{\label{ch:evaluation:lengths}Introns of Various Lengths}

Figure \ref{fig:evaluation:donor-intron-length-error} and Figure
\ref{fig:evaluation:acceptor-intron-length-error} show the dependency of the
error rates of donor and acceptor models, respectively, on the splice sites
associated with introns of various lengths. The figures demonstrate that the
models are systematically not recognizing the splice sites of introns shorter
than 40 nucleotides. The error rate is the smallest in the area around 55
nucleotides and goes up for splice sites of longer introns. Low performance on
short introns is more pronounced in the acceptor model.

The effect could be, in part, explained by the distribution of intron lengths
in the data, see Figure \ref{fig:data:intron-len-dist}. Training samples of the
models were randomly drawn from the dataset of training organisms. Therefore,
the lengths of most of the introns associated with the positive splice sites
the model ``saw'' during training were concentrated around 55 nucleotides.

Introns having lengths shorter than 30 nucleotides are extremely rare in the
gene databases across all eukaryotic organisms
\cite{piovesan2015identification}. Their existence in the data might be a
result of incorrect annotation, and it is hypothesized that they do not exist
in nature due to the minimum sequence elements needed for their splicing
\cite{piovesan2015identification}. Introns shorter than 30 nucleotides are
rare, but they are present in the data used in this work. In light of this, the
high error rate on the splice sites of very short introns could be caused by
data quality.

\begin{figure}
  \centering
  \includegraphics[width=0.8\textwidth]{figures/donor_evaluation/intron-lenght-error.pdf}
  \caption{Dependence of the NN\,400 donor model error rate on positive splice
    site samples and the lengths of associated introns.}
  \label{fig:evaluation:donor-intron-length-error}
\end{figure}

\begin{figure}
  \centering
  \includegraphics[width=0.8\textwidth]{figures/acceptor_evaluation/intron-lenght-error.pdf}
  \caption{Dependence of NN\,400 acceptor model error rate on positive splice
    site samples and lengths of associated introns.}
  \label{fig:evaluation:acceptor-intron-length-error}
\end{figure}

\section{\label{ch:evaluation:correlation}Donor and Acceptor Model Correlation}

Figure \ref{fig:evaluation:donor-acceptor-dependency} depicts the output
dependency of the donor splice site model and the acceptor splice site model
when predictions are performed on the (opposite) splice sites of the same
intron. Per organisms correlations of the outputs are given in Table
\ref{table::evaluation:donor-acceptor-dependency}.

The correlation is negative only for Rozal\_SC1 and Enche1; it is larger than
$0.6$ for all other test organisms. This might be related to the fact that the
performance on the organisms Rozal\_SC1 and Enche1 is lower by a big margin
compared to all the other test organisms, see Table
\ref{table:evaluation:donor} and Table \ref{table:evaluation:acceptor}.

\begin{table}
  \begin{center}
    \begin{tabular}{ | l | d{2} | }
      \hline

      \multicolumn{1}{| l |}{\textbf{Organism ID}} &
      \multicolumn{1}{| l |}{\textbf{Correlation}} \\

      \hline
      Aspwe1 & 0.69 \\
      Mycalb1 & 0.63 \\
      Allma1 & 0.73 \\
      Chytri1 & 0.65 \\
      Rozal\_SC1 & -0.05 \\
      Enche1 & -0.13 \\
      Liccor1 & 0.64 \\
      Coere1 & 0.66 \\
      \hline
    \end{tabular}
  \end{center}
  \caption{\label{table::evaluation:donor-acceptor-dependency}The Pearson
    correlation coefficient of NN\,400 donor and NN\,400 acceptor model outputs
    on the splice sites of the same intron.}
\end{table}

\begin{figure}
  \centering
  \includegraphics[width=0.8\textwidth]{figures/donor-acceptor-dependency.pdf}
  \caption{Dependency of NN\,400 donor and NN\,400 acceptor model outputs on
    the splice sites of the same intron.}
  \label{fig:evaluation:donor-acceptor-dependency}
\end{figure}

The data illustrates that an intron with a ``hard-to-recognize'' splice site is
of diminished ``visibility'' even for the model of the opposite splice site. In
Section \ref{ch:evaluation:sensitivity}, it is demonstrated that both the donor
splice site model and the acceptor splice site model are sensitive in the
region of the opposite splice site---this likely explains the correlations.

If donor and acceptor models were independent, the probability of the detection
of an intron would be a multiplication of the probabilities of its splice sites
being detected (independently). With strong correlation of donor and acceptor
model outputs, the overall likelihood of the detection of a true intron is much
higher. This is shown by Table \ref{table::evaluation:both-splice-sites} that
gives a percentage of introns with both splice sites detected by the real
models and by hypothetical statistically independent models.

\begin{table}
  \begin{center}
    \begin{tabular}{ | l | d{2} | d{2} | }
      \hline

      \multicolumn{1}{| l |}{\textbf{Organism ID}} &
      \multicolumn{1}{| l |}{\textbf{Detected Introns}} &
      \multicolumn{1}{| l |}{\textbf{Multiplication}} \\

      \hline
      Aspwe1     & 91.6\% & 88.0\% \\
      Mycalb1    & 72.8\% & 65.2\% \\
      Allma1     & 78.0\% & 69.8\% \\
      Chytri1    & 79.2\% & 72.7\% \\
      Rozal\_SC1 & 12.5\% & 13.6\% \\
      Enche1     & 42.1\% & 42.1\% \\
      Liccor1    & 92.0\% & 89.2\% \\
      Coere1     & 51.1\% & 37.2\% \\
      \hline
    \end{tabular}
  \end{center}
  \caption{\label{table::evaluation:both-splice-sites}The percentage of introns
    whose splice sites were recognized by both the NN\,400 donor model and the
    NN\,400 acceptor model and the percentage of detected introns, if the
    models were statistically independent but with the same true positive
    rate.}
\end{table}

\section{\label{ch:evaluation:whole}Whole Gene Prediction Pipeline}

As discussed in Chapter \ref{ch:intro} and Chapter \ref{ch:motivation}, the
primary purpose of splice site detection models and their combination to the
full intron detection pipeline is to improve the performance of a larger gene
prediction pipeline, which is utilized to detect gene homologies in novel and
known fungal DNA sequences in metagenomes.

Utilization of the intron detection pipeline based on SVM, as developed in
\cite{barucic}, has significantly improved the gene prediction by increasing
the number of discovered gene homologies when the gene prediction pipeline was
executed separately with and without the intron detection.

Based on the results with SVM models and better performance of the neural
networks compared to SVM, it is expected that the gene prediction will further
benefit with the introduction of the neural network-based intron detection
pipeline.

\chapter{\label{ch:conclusion}Conclusion}

\minitoc

The goal of this work was to develop an algorithm for automated intron
detection in fungal metagenomes with the use of neural networks. Emphasis was
put on computational requirements and comparison with an existing intron
detection pipeline based on support vector machines (SVM).

The resulting pipeline contains two splice site classification models based on
deep recurrent convolutional neural networks. As opposed to the pipeline based
on SVM, no third intron model is used. The solution outperforms the approach
based on SVM and requires 46 times less computational resources for
classification. The neural networks also generalize in a better way than the
SVM models, and therefore, only one model is used for all phyla.

Up to 91.6\% introns on the Ascomycota phylum and 72.8\% introns on the
Basidiomycota phylum are detected with the neural network-based pipeline.

\section{Future Work}

This work opens a possibility to implement an online, web-based service for
intron detection and/or removal. This could be done as a simple webpage where
the user uploads a FASTA file and gets a GFF file with automatically annotated
introns within minutes. All computations could easily be distributed thanks to
the design of the pipeline and characteristics of the data. Low CPU usage of
the developed classification models makes this possibility economically
feasible, see Chapter \ref{ch:evaluation}.

A direction for further investigation is the generalization of the developed
models. Interesting results could be obtained by evaluating the model
performance on different kingdoms and analyzing the errors it makes. This
raises two questions: how well and consistently the model performs when trained
and evaluated on distant organisms and whether the same model architecture and
approach works well on different kingdoms.

Using the pre-trained models and utilizing transfer learning on organisms with
a low amount of annotated data is also a possibility for future investigation.

Using the models in quality assurance procedures of non-automated genome
annotation is also an interesting topic worth further exploration. Paying more
attention to the annotations that are in disagreement with the model
predictions might improve the overall data quality with fixed effort spent.

Performing in vitro or in vivo experiments might be laborious, time-consuming,
and expensive. Using the developed in silico methods before the ``wet''
research is started might save resources, time, and help future research to
focus on promising areas. This is yet another area of potential future
research.


%% APPENDICES
% Starts lettered appendices, adds a heading in table of contents, and adds a
% page that just says "Appendices" to signal the end of your main text.
% \startappendices
% Add or remove any appendices you'd like here:
% \include{text/appendix-1}

%%%%% REFERENCES

\begin{savequote}[8cm]
Biodiversity is our most valuable but least appreciated resource. \qauthor{---
  Edward O. Wilson}
\end{savequote}

\setlength{\baselineskip}{0pt} % JEM: Single-space References

{\renewcommand*\MakeUppercase[1]{#1}%
\printbibliography[heading=bibintoc,title={\bibtitle}]}


\end{document}
