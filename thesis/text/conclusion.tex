\chapter{\label{ch:conclusion}Conclusion}

\minitoc

The goal of this work was to develop an algorithm for automated intron
detection in fungal metagenomes with the use of neural networks. Emphasis was
put on computational requirements and comparison with an existing intron
detection pipeline based on support vector machines (SVM).

The resulting pipeline contains two splice site classification models based on
deep recurrent convolutional neural networks. As opposed to the pipeline based
on SVM, no third intron model is used. The solution outperforms the approach
based on SVM and requires 46 times less computational resources for
classification. The neural networks also generalize in a better way than the
SVM models, and therefore, only one model is used for all phyla.

Up to 91.6\% introns on the Ascomycota phylum and 72.8\% introns on the
Basidiomycota phylum are detected with the neural network-based pipeline.

\section{Future Work}

This work opens a possibility to implement an online, web-based service for
intron detection and/or removal. This could be done as a simple webpage where
the user uploads a FASTA file and gets a GFF file with automatically annotated
introns within minutes. All computations could easily be distributed thanks to
the design of the pipeline and characteristics of the data. Low CPU usage of
the developed classification models makes this possibility economically
feasible, see Chapter \ref{ch:evaluation}.

A direction for further investigation is the generalization of the developed
models. Interesting results could be obtained by evaluating the model
performance on different kingdoms and analyzing the errors it makes. This
raises two questions: how well and consistently the model performs when trained
and evaluated on distant organisms and whether the same model architecture and
approach works well on different kingdoms.

Using the pre-trained models and utilizing transfer learning on organisms with
a low amount of annotated data is also a possibility for future investigation.

Using the models in quality assurance procedures of non-automated genome
annotation is also an interesting topic worth further exploration. Paying more
attention to the annotations that are in disagreement with the model
predictions might improve the overall data quality with fixed effort spent.

Performing in vitro or in vivo experiments might be laborious, time-consuming,
and expensive. Using the developed in silico methods before the ``wet''
research is started might save resources, time, and help future research to
focus on promising areas. This is yet another area of potential future
research.
