\chapter{\label{ch:rcnn}Recurrent Convolutional Neural Networks}

\minitoc

This work aims to develop a method based on deep neural networks which would
detect and remove introns from the DNA sequences of various fungal organisms.

Removal of introns is done in these steps:

\begin{enumerate}
  \item all candidate donor and acceptor splice sites are identified based on
    consensus dinucleotides,
  \item these candidate splice sites are classified as true and false splice
    sites with separate donor and acceptor recurrent neural networks,
  \item positive classifications are combined to form candidate introns,
  \item candidate introns are assigned a score equal to the multiple of the
    respective donor and acceptor splice site model confidence,
  \item and overlapping candidate introns with non-highest score are filtered
    out.
\end{enumerate}

Section \ref{ch:rcnn:overview} gives an overview of the neural networks used
and the overall pipeline. Section \ref{ch:rcnn:candidates} talks about the
selection of candidate splice sites---i.e. locations in source DNA sequences
(from a metagenome) which are considered as potential donors and acceptors.
These locations are then classified with the splice site networks. Section
\ref{ch:rcnn:encoding} describes how the DNA sequences are encoded in matrices
so that they could be used as inputs to the neural networks. Section
\ref{ch:rcnn:architecture} talks about the architecture and other aspects of
the splice site classification neural network. Section \ref{ch:rcnn:criteria}
defines the criteria used during network selection. Finally, Section
\ref{ch:rcnn:splice-site-training} describes the training of the neural
networks.

\section{\label{ch:rcnn:overview}Overview}

All critical sections in the aforementioned intron detection and deletion of
the pipeline---i.e. two neural networks---are created as part of this work. An
existing pipeline from work \cite{barucic} (which originally used the support
vector machines in the critical section) was adapted to obtain a complete
pipeline.

Two separately trained neural networks with equal architecture were used for
the classification of donor and acceptor splice sites within original DNA
sequences (i.e. in full DNA sequence and before transcription to RNA). To
accompany the complexity of all regulatory and other sequences involved in
intron splicing, convolutional layers were utilized in target neural networks.
Recurrent layers with long short-term memory (LSMT) units were used to enable
the detection of partially shift invariant inter-dependencies and redundancies
between non-adjacent parts of the intron sequences.

The neural networks were trained to distinguish between true and false splice
sites. The input to the neural network is a window spanning some distance to
both sides around the candidate splice site. The window size has been selected
to be large enough so that the network can detect all important features within
the DNA sequence. The output of the network is the confidence of the input
sample being a true splice site.

The pipeline reported in \cite{barucic} requires a third model which further
filters candidate introns. The need for this additional step was motivated by
the high false positive rate of splice site classification models. In this
work, the additional filtering step was assessed as counterproductive because
splice site classification models based on neural networks have a smaller false
positive rate and further filtering of introns would impact the overall intron
recall. See Section \ref{ch:evaluation:comparison} for evaluation metrics.

\section{\label{ch:rcnn:candidates}Candidate Splice Site Selection}

Only splice sites with the consensus dinucleotides were used during the
training, evaluation, and testing of the networks. These are \Verb_GT_ for
donor and \Verb_AG_ for acceptor splice sites. This decision was made because
non-consensus splice sites are rare, and therefore, there was not enough
training and evaluation data to be included. In fact, its inclusion would
drastically increase the number of false positive classifications. It omission
has only a minuscule negative effect on the number of false negative
classifications. See Section \ref{ch:data:stats} which contains more related
statistics.

False splice sites included in training, validation, and test datasets are
selected only from the gene areas in source DNA sequences. See Section
\ref{ch:automation:overview} for the definition of ``gene'' used in this
thesis. Filtering to genes is motivated by the fact that intra-genetic DNA does
not play an important role during sequence alignment and homologous gene search
in the larger gene prediction pipeline. Therefore, removal of presumed introns
from these intra-genetic areas should not have a large impact on overall
results. Only the ability to recognize false splice site within genetic areas
was assumed to play an important role. Furthermore, sequences which would be
biologically processed as introns might be present in intra-genetic areas and
are therefore labeled as negative examples. This would lead to confusion of the
trained network as well as a worse performance.

\section{\label{ch:rcnn:encoding}Sequence Encoding}

Splice site classification neural networks were trained to map an input
sequence window to a value between 0 and 1. The sequence is encoded as a matrix
of dimensions $N \times 5$, where the $N$ rows represent relative positions
along the DNA sequence and the columns represent one-hot-encoded nucleotides.
Various input window sizes $N = N_{upstream} + N_{downstream}$ were evaluated,
see Table \ref{table:rcnn:win-size-donor} and Table
\ref{table:rcnn:win-size-acceptor}. Each nucleotide is encoded as a
5-dimensional vector, with value $1.0$ at the dimension of the represented
nucleotide and values $0.0$ at other dimensions. The first four dimensions
represent the nucleotides adenine (A), thymine (T), cytosine (C), and guanine
(G). The fifth dimension represents ``any symbol'' (usually missing or corrupt
data).

\begin{equation}\label{ch:rcnn:ex-donor}
  \begin{bmatrix}
    1 & 0 & 0 & 0 & 0 \\  % A
    0 & 0 & 0 & 1 & 0 \\  % G
    0 & 1 & 0 & 0 & 0 \\  % T
    0 & 0 & 0 & 0 & 1 \\ % N
    1 & 0 & 0 & 0 & 0 \\ % A
    1 & 0 & 0 & 0 & 0  % A
  \end{bmatrix}
\end{equation}

Matrix \ref{ch:rcnn:ex-donor} illustrates a six-nucleotide long,
one-hot-encoded DNA sequence window that reads AGTNAA, which contains the
consensus donor dinucleotide GT at position 1.

\section{\label{ch:rcnn:architecture}Neural Network Architecture}

This section reports selected splice site classification network architecture,
see Figure \ref{fig:rcnn:architecture} and explains the theoretical background
and motivation behind the input and each selected layer, as well as the
empirical findings.

\begin{figure}
  \centering
  \includegraphics[width=\textwidth,height=0.9\textheight]{figures/architecture.pdf}
  \caption{Model architecture visualization}
  \label{fig:rcnn:architecture}
\end{figure}

Table \ref{table:rcnn:win-size-donor} and Table
\ref{table:rcnn:win-size-acceptor} show the dependency between the shape of the
input window to the neural network and its performance. The tables clearly
demonstrate that the window overlap with the associated intron -- located at
the downstream direction for donors and upstream direction for acceptors --
plays a dominant role and that classifications performed on windows with no
overlap with the associated intron have very poor performance. This finding is
in agreement with the biological understanding of splicing, which is mostly
dependent on DNA sequences inside the intron. See Section
\ref{ch:background:dna-rna} which covers the biological background and Section
\ref{ch:evaluation:sensitivity} which explores the sensitivity of the trained
network at various input positions.

\begin{table}
  \begin{center}
    \begin{tabular}{ | l | d{2} | d{2} | d{2} | d{2} | d{2} | d{2} | }
      \hline

      &
      \multicolumn{2}{| l |}{\textbf{0}} &
      \multicolumn{2}{| l |}{\textbf{100}} &
      \multicolumn{2}{| l |}{\textbf{200}} \\

      \hline

      &
      \multicolumn{1}{| l |}{\textbf{Precision}} &
      \multicolumn{1}{| l |}{\textbf{Recall}} &
      \multicolumn{1}{| l |}{\textbf{Precision}} &
      \multicolumn{1}{| l |}{\textbf{Recall}} &
      \multicolumn{1}{| l |}{\textbf{Precision}} &
      \multicolumn{1}{| l |}{\textbf{Recall}} \\

      \hline
      \textbf{0} & \multicolumn{1}{| c |}{-} & \multicolumn{1}{| c |}{-} & 51.8\% & 85.2\% & 52.2\% & 85.4\% \\
      \hline
      \textbf{100} & 14.3\% & 32.2\% & 62.4\% & 87.8\% & 63.3\% & 88.0\% \\
      \hline
      \textbf{200} & 14.6\% & 34.0\% & 63.6\% & 88.1\% & 64.6\% & 87.9\% \\
      \hline
    \end{tabular}
  \end{center}
  \caption{\label{table:rcnn:win-size-donor}Dependency between classification
    performance on donors and the number of nucleotides upstream (rows) and
    downstream (columns) from the candidate splice site included in the input
    window.}
\end{table}

\begin{table}
  \begin{center}
    \begin{tabular}{ | l | d{2} | d{2} | d{2} | d{2} | d{2} | d{2} | }
      \hline

      &
      \multicolumn{2}{| l |}{\textbf{0}} &
      \multicolumn{2}{| l |}{\textbf{100}} &
      \multicolumn{2}{| l |}{\textbf{200}} \\

      \hline

      &
      \multicolumn{1}{| l |}{\textbf{Precision}} &
      \multicolumn{1}{| l |}{\textbf{Recall}} &
      \multicolumn{1}{| l |}{\textbf{Precision}} &
      \multicolumn{1}{| l |}{\textbf{Recall}} &
      \multicolumn{1}{| l |}{\textbf{Precision}} &
      \multicolumn{1}{| l |}{\textbf{Recall}} \\

      \hline
      \textbf{0} & \multicolumn{1}{| c |}{-} & \multicolumn{1}{| c |}{-} & 10.2\% & 22.1\% & 10.4\% & 24.3\% \\
      \hline
      \textbf{100} & 53.1\% & 86.3\% & 57.7\% & 87.2\% & 56.6\% & 87.2\% \\
      \hline
      \textbf{200} & 56.1\% & 86.4\% & 58.6\% & 87.3\% & 58.8\% & 87.4\% \\
      \hline
    \end{tabular}
  \end{center}
  \caption{\label{table:rcnn:win-size-acceptor}Dependency between
    classification performance on acceptors and the number of nucleotides
    upstream (rows) and downstream (columns) from the candidate splice site
    included in the input window.}
\end{table}

The convolutional layer allows for sparse connectivity between successive
layers and parameter sharing by applying the same kernel parameters to a
limited receptive field \cite[p.~330]{goodfellow2016deep}. This leads to
drastic reduction of the number of trainable parameters which, in turn,
decreases the computing time and needed training dataset size. Convolutions are
equivariant to translation \cite[p.~334]{goodfellow2016deep} and are therefore
theoretically capable of learning various features at different relative
positions inside input DNA sequence windows from fewer samples. It is
hypothesized that such position-independent features exist in the data, which
motivated the use of convolutional layers in the resulting network
architecture. It has been empirically verified that convolutional layers
improved various network performance metrics at low additional computational
costs.

Leaky rectified linear activation functions (Leaky ReLU) were used throughout
the network. Use of the ReLU activation function and its variations has many
benefits. ReLU was among the most frequently used and successful activation
function as of 2017 \cite{ramachandran2017searching}; it leads to extensive
research on them. Networks using ReLU units are more easily trained because
they do not have problems with vanishing and exploding gradients
\cite{ramachandran2017searching}. Leaky ReLU was used instead of plain ReLU
because it provides similar computational complexity but avoids 0 gradient
\cite{maas2013rectifier}, thereby leading to improved performance.

Skip connections forming residential networks (ResNet) are introduced in the
convolutional part of the network. Deep networks may suffer from the vanishing
gradient and the degradation problem, which could be resolved by using ResNet
\cite{he2016deep}.

Recurrent neural networks (RNNs) are capable of recognizing same features at
various positions in the input sequence and even recognize their repetition and
mutual dependence \cite[p.~367]{goodfellow2016deep}. To some extent, it is
possible to achieve similar position-independent feature recognition with
convolutional layers. However, their processing is shallower and disallows
greater interdependence between different positions in the sequence
\cite[p.~368]{goodfellow2016deep}. As of 2016, gated RNNs, which include
networks with long short-term memory (LSTM) units, were the most effective type
of RNNs \cite[p.~404]{goodfellow2016deep}. LSTM units allow the retention of
information over large number of time steps without any problems with vanishing
or exploding gradients \cite[p.~404]{goodfellow2016deep}. Other works also
report the successful use of RNN layers for splice site detection
\cite{lee2015dna}. For these reasons, a bidirectional RNN layer was used in the
selected architecture.

The work \cite{lee2015dna} reports no improvement after adding more recurrent
layers. The same finding of no statistically significant improvement after
adding another recurrent layer was observed in this work.

The used recurrent layer produces output at each step. This decision has been
motivated by the need to distinguish true splice sites that are located exactly
in the middle of the input sequence from those located in close proximity in
the middle. Bidirectional RNN was used to allow the recognition of both
downstream and upstream dependencies.

The aforementioned multilayered architecture allows for the learning of
complex, highly non-linear mappings between input DNA sequences and output
splice site confidences. The network has 49\,921 learnable parameters, which
give large learning capacity. However, neural networks with large capacities
tend to overfit the training data \cite{srivastava2014dropout}. This effect was
empirically observed and confirmed during experiments on neural network
architectures without sufficient regularization techniques. A dropout layer was
successfully used in the final architecture to completely overcome overfitting.

Dropout is a technique of omission of a given fraction of neural units in a
given layer during training. A different stochastic unit selection is done
during each training mini-batch. Classical optimization based on
backpropagation is then applied to this reduced network
\cite{srivastava2014dropout}. All neurons are applied with a scaling factor
during the later inference of a trained network. Dropout is an approximation of
an equally weighted geometric mean of the predictions of multiple neural
networks with shared parameters \cite{srivastava2014dropout}. The number of
dropouts and therefore the number of synthetic networks are exponential with
the number of neural units \cite{srivastava2014dropout}. This technique is
highly efficient from the point of view of computational resources and prevents
overfitting \cite{srivastava2014dropout}.

\section{\label{ch:rcnn:criteria}Model Criteria}

Model selection was based on the following criteria:

\begin{itemize}
  \item ability of the model to generalize and perform consistently among
    different fungal organisms,
  \item good performance metrics, namely its precision and recall,
  \item model complexity and computational intensity during inference.
\end{itemize}

The generalization capability of the model was an important aspect because the
model would be applied to a wide variety of previously unknown fungal genomes.
For this reason, all experimental models were evaluated on individual organisms
separately as well as on larger sets of organisms. Consistency was taken into
account in this regard.

Precision and recall were used to evaluate the performance of the models. Small
false positive rate was emphasized during the development of the network
architecture and the selection of binary classification threshold values. This
is because consensus dinucleotides are more than an order of magnitude more
frequent in target DNA sequences than true splice sites, see Section
\ref{ch:data:stats} for more details.

Final production computational resource utilization was also an important
consideration during the design of the network architecture and pre-processing
and post-processing procedures. This aspect is important because metagenomes on
which they will be applied are enormous and might go well beyond $10^9$
nucleotides. The possible use of the results of this work in an automated
online annotation tool is another consideration, and high computational
requirement would make such a goal economically unfeasible. This criterion has
led to the selection of a sub-optimal network from the point of view of
accuracy as networks with larger complexity and window size had slightly better
results. But the improvement was not significant and would lead to a large
increase in computational requirements.

\section{\label{ch:rcnn:splice-site-training}Neural Network Training}

Stochastic gradient descent (SGD) was used for the optimization during network
training. Other optimization techniques, such as the adaptive methods AdaGrad,
RMSProp, or Adam, were also tested, but SGD yielded superior results. It has
been reported that adaptive methods converge faster during the initial phases
of training; but given enough training time, it may lead to worse
generalization \cite{wilson2017marginal}. This effect was empirically confirmed
in this work.

The initial learning rate was set to $\alpha = 0.01$ and multiplied by the
factor of $0.2$ after every epoch which did not lead to a decrease in
validation loss. The training was automatically stopped after 10 successive
epochs with no significant decrease in validation loss. The model with the
lowest after-epoch validation loss was used during successive evaluation and
tests.

Binary cross-entropy was chosen as a surrogate loss function.

It has been reported that small batch sizes ($m \le 32$) lead to better test
results and generalization. For some networks and tasks, this effect might go
down to the batch size as small as $m = 2$ \cite{masters2018revisitingb}. The
batch size of $m = 16$ was used during training.

\begin{figure}
  \centering
  \includegraphics[width=\textwidth]{figures/donor_evaluation/training-loss.pdf}
  \caption{Training and validation loss during training of the donor model.
    Locally weighted scatterplot smoothing (LOWESS) is used as a local
    regression of the training loss.}
  \label{fig:rcnn:training-loss}
\end{figure}

Source data comprises of 862 organisms, see Section \ref{ch:data:taxonomy}.
These were randomly split into training (769 or 90\%), validation (85 or 9\%)
and test (8 or 1\%) datasets. See Section \ref{ch:evaluation:datasets} for more
information on the test dataset.

Over three million samples were used in the training dataset and over 85\,000
samples were used in the validation dataset. The number of samples included in
the validation dataset is 1000 per organism and is greater than the minimum
required number of 75\,759 samples as derived in Section
\ref{ch:rcnn:dataset-size}.

Per organism stratified random sampling was used for the creation of the
datasets. Donor and acceptor samples were selected independently, i.e. some
introns might be only half-included.

The training dataset comprises of 30\% of positive samples and the validation
dataset comprises of 50\% of positive samples. Overrepresentation of the
negative samples in the training dataset was selected to achieve low false
positive rate. This is motivated by the low ratio of the number of true splice
sites to the number of false splice sites in the real-world data. See Section
\ref{ch:data:stats} which reports various data statistics.

After each epoch, training samples were randomly reshuffled.
Training and validation loss progression is illustrated by Figure
\ref{fig:rcnn:training-loss}.

Keras \cite{chollet2015keras} with TensorFlow \cite{abadi2016tensorflow}
backend was used for training and prediction. Preprocessed data in the form of
Numpy NPZ files with input matrices and desired outputs is used during
training, see Section \ref{ch:automation:samples}. Data was loaded gradually
from the disk and not kept in memory because of the large number of samples
used during the training.

See Chapter \ref{ch:evaluation} for detailed evaluation of achieved results.

\section{\label{ch:rcnn:dataset-size}Dataset Size}

The validation dataset always consisted of 50\% of negative samples and 50\% of
positive samples. The validation dataset size $l$ was chosen so that the
probability of empirical true positive rate (TPR) or empirical true negative
rate (TNR) being more than $\epsilon$, different from the true TPR or the true
TNR in any of $n$ experiments, was smaller or equal to $R$. See Section
\ref{ch:background:evaluation} for more information on evaluation.

Formula \ref{eq:rcnn:proba-single} gives the maximum probability $R_1$ of
seeing a “bad” TPR or TNR measurement in a single evaluation for the above
condition to hold. This is derived from the fact that $2 \cdot n$ measurements
need to be performed to obtain empirical TPR and empirical TNR for $n$
different model setups.

Formula \ref{eq:rcnn:hoeffding} is derived from Hoeffding's inequality
\cite{hoeffding1994probability} and gives the minimum dataset size $l$. The
right part of the equation is multiplied by 2 because the dataset is split into
half between positive and negative samples.

A dataset comprising $l = 75759$ samples is needed for $n = 25$, $R = 0.05$
(5\%) and $\epsilon = 0.01$ (1\%).

\begin{equation}
  R_1 = 1 - (1 - R)^\frac{1}{2 \cdot n}
  \label{eq:rcnn:proba-single}
\end{equation}

\begin{equation}
  l = \ceil*{2 \cdot \frac{\log 2 - \log R_1}{2 \cdot \epsilon^2}}
  \label{eq:rcnn:hoeffding}
\end{equation}
