\chapter{\label{ch:background}Background}

\minitoc

The nucleic DNA molecules or chromosomes of Eukaryotic organisms play many
roles in cell biology. Different positions (loci) in a chromosome play
different, often overlapping and complex, roles \cite{pennisi2007dna}. DNA
functions and its structure are not yet fully understood
\cite{slijepcevic2018genome}.

The previously mentioned complexity opens up the possibility of novel
approaches like state-of-the-art machine learning techniques in order to study
the DNA structure, its roles and functions, and the detection of already known
parts in new DNA sequences. For example, it has been shown that feed-forward
neural networks can represent a wide variety of functions
\cite{cybenko1989approximation}. The versatility and effectiveness of
artificial neural networks has been practically demonstrated in many fields.
Examples of this are general game play \cite{silver2016mastering}, face
recognition \cite{sun2015deepid3}, speech recognition
\cite{xiong2016achieving}, or medical image classification
\cite{bychkov2018deep}, among others.

\section{\label{ch:background:dna-rna}DNA and RNA Structure}

Polynucleotide is a linear chain of up to 20 different nucleotide monomers
joined by the phosphodiester (covalent) bond
\cite[p.~308,p.~347]{king2006dictionary}. Deoxyribonucleic acid (DNA) and
ribonucleic acid (RNA) are two types of polynucleotides that are abundant in
natural life. Both have important biological functions. DNA is composed of
individual nucleotides Adenine (A), Thymine (T), Cytosine (C), and Guanine (G)
\cite[p.~4,~p.~20,~p.~107]{pollard2016cell}. RNA is made from nucleotides
Adenine (A), Uracil (U), Cytosine (C), and Guanine (G)
\cite[p.~11]{jankowski1996clinical}. RNA molecules first appeared around four
billion years ago as a first form of life \cite[p.~412]{king2006dictionary}.

The DNA molecule is directional due to asymmetry of individual nucleotides
\cite[p.~42]{pollard2016cell}. See Figure \ref{fig:intro:nucleotide}. Based on
The chemical convention of naming carbon atoms in the nucleotide sugar-ring,
one side of DNA is called 5\textquotesingle{}-end and the other side
3\textquotesingle{}-end. DNA and RNA are synthesized in the 5\textquotesingle{}
to 3\textquotesingle{} direction \cite[p.~167,~p.~728]{pollard2016cell}. When
referring to relative positions in a DNA sequence, upstream and downstream
refer to the 5\textquotesingle{} and the 3\textquotesingle{} directions
respectively. Similarly, by convention, DNA sequences are usually written and
stored in the 5\textquotesingle{} to 3\textquotesingle{} direction, unless
explicitly stated or needed otherwise.

Chromosomes are enormous DNA molecules which encode the majority of genetic
information in fungi and other kingdoms of species. DNA in chromosomes consists
of two coiled chains of polynucleotide strands forming a double helix.

\begin{figure}
  \centering
  \includegraphics[width=0.3\textwidth]{figures/nucleotide.pdf}
  \caption{Molecular structure of a nucleotide \cite{nucleotide-img}}
  \label{fig:intro:nucleotide}
\end{figure}

Different parts of a chromosome have different functions in a cell. Within
chromosomes are continuous parts which form genes, which are nucleotide
sequences that encode gene products---either RNA or protein. Genes have a
complex structure that contains regulatory sequences, exons, introns, and other
areas. The regulatory sequences of a gene, located at the extremities of the
gene, contain a promoter at the 5\textquotesingle{} side and a terminator at
the 3\textquotesingle{} side of the gene. The promoter and terminator mark the
beginning and end of the transcribed region of the gene respectively.

In eucaryotic organisms, gene transcription forms a primary transcript,
alternatively called precursor mRNA or Pre-mRNA, which consists of exons and
introns and lacks 5\textquotesingle{} cap and poly(A) tail. 5\textquotesingle{}
cap and poly(A) tail are added at later stages of gene expression
\cite{cooper2000cell}. Afterward, introns are spliced out during the
post-transcriptional modification and the remaining exons form a final mature
mRNA that, in some cases, encodes protein. Both ends of mRNA contain
untranslated regions (UTR) 5\textquotesingle{}-UTR and 3\textquotesingle{}-UTR
enclosing the final protein coding region \cite{shafee2017eukaryotic}. The gene
structure and processing are illustrated in Figure
\ref{fig:intro:gene-structure}.

\begin{figure}
  \centering
  \includegraphics[width=\textwidth]{figures/gene-structure.pdf}
  \caption{Gene structure and processing \cite{shafee2017eukaryotic}}
  \label{fig:intro:gene-structure}
\end{figure}

Within introns, the donor splice site, branch point, and the acceptor splice
site are used for splicing. The donor splice site lies at the
5\textquotesingle{}-end of the intron, the acceptor splice site lies at the
3\textquotesingle{}-end of the intron, and the branch point lies 18--40
nucleotides upstream from the acceptor site \cite{clancy2008rna}. In the great
majority of the cases, the 5\textquotesingle{}-end of an intron starts with
highly preserved \Verb_GU_ nucleotides and the 3\textquotesingle{}-end of an
intron ends with \Verb_AG_ nucleotides. The highly preserved dinucleotides,
which make first two symbols in an intron (at 5\textquotesingle{}-end) and last
two symbols in an intron (at 3\textquotesingle{}-end), are commonly referred to
as consensus dinucleotides. The branch point always contains adenine, but it is
otherwise more variable \cite{clancy2008rna}. Cell's splicing machinery is
called spliceosomes and removes introns in a sequence of complex steps. The
exact end-to-end mechanisms are not yet fully understood \cite{clancy2008rna}.

\section{\label{ch:background:alignment}Biological Sequence Alignment and Search}

Sequence alignment is an algorithm which searches for the best scoring
alignment of two or more sequences of symbols by inserting gaps into any of the
sequences. The resulting alignment contains indels (insertions and deletions),
mismatches, and matches. The alignments are usually scored by summing penalties
for indels with (mis)match scores at other positions. The (mis)match scores are
taken from a substitution matrix and the gaps are frequently scored with a
gap-open penalty summed up with a penalty proportional to the gap length. An
example pairwise alignment of two sequences is given in Figure
\ref{fig:background:global-alignment}.

\begin{figure}
\centering
\begin{BVerbatim}[baselinestretch=0.75]
A-TATCATGA
AGTA-CATGG
\end{BVerbatim}
\caption{Global sequence alignment example}
\label{fig:background:global-alignment}
\end{figure}

Two types of alignments exist: global and local. A global alignment produces
sequences of equal length, while a local alignment produces an alignment only
of parts of the sequences. See Figure \ref{fig:background:local-alignment} with
an example of a local pairwise alignment.

\begin{figure}
\centering
\begin{BVerbatim}[baselinestretch=0.75]
ATTATCATGA
  TA-CA
\end{BVerbatim}
\caption{An example of a sequence (second row) locally aligned to a larger
  sequence (first row)}
\label{fig:background:local-alignment}
\end{figure}

There exist exact algorithms like the Needleman-Wunsch algorithm for global
alignments \cite{needleman1970general} and the Smith-Waterman algorithm for
local alignments \cite{smith1981identification}. The exact algorithms have
large asymptotic time and memory complexity. An optimized version of the
Needleman-Wunsch algorithm has $\mathcal{O}(mn / \log(n))$ asymptotic time
complexity, where $m$ and $n$ are lengths of the sequences
\cite[p.~35]{sung2009algorithms}. The asymptotic time complexity of the
Smith-Waterman algorithm is $\mathcal{O}(mn)$ \cite[p.~40]{sung2009algorithms}.
This makes their usage for search in large databases unfeasible because their
asymptotic time has to be further multiplied by the number of database entries.

For reasons of efficiency, heuristic sequence alignment algorithms are in wide
use. These algorithms do not guarantee optimal results but have a much shorter
execution time and smaller memory usage. Among the heuristic algorithms is
BLAST (basic local alignment search tool), one of the most used algorithms for
sequence searching \cite{casey2005blast}.

E-value is the expected number of coincidental matches in the database of a
given size with the match score equal or greater than the found score
\cite[p.~119]{sung2009algorithms}. In other words, a large E-value signifies a
high probability that a match is only coincidental.

When searching for a sequence in a database, a local alignment, such as BLAST,
is performed against all available sequences. All matches with a high enough
score giving a sufficiently small E-value are reported as results.

\section{Neural Networks}

The artificial neural network is a network of interconnected artificial
neurons. The activation $y_q$ (output) of a simple artificial neuron, $q$, can
be described with Formula \ref{eq:intro:neuron-activation}, where $g:
\mathbb{R} \rightarrow \mathbb{R}$ is the neuron activation function, $x_i$ is
either one of the neural network inputs or the activation of the preceding
neuron $i$, $w_{iq} \in \mathbb{R}$ is the weight of the connection from neuron
$i$ to neuron $q$, and $b_q \in \mathbb{R}$ is the bias of neuron $q$. The
weights and biases are learned during ANN training.

\begin{equation}
  y_q = g\left(\sum(x_i \cdot w_{iq}) - b_q\right)
  \label{eq:intro:neuron-activation}
\end{equation}

The architecture of a neural network consists of the connections between
individual neurons and activation functions, which are specified as
meta-parameters \cite[p.~193]{goodfellow2016deep}. Most neural networks could
be organized into layers. The neurons in each layer connect only to neurons
from preceding layers \cite[p.~193]{goodfellow2016deep}.

A neural network is considered deep if it consists of more than three levels of
compositions of non-linear operations \cite[p.~6]{bengio2009learning}.
Therefore, a neural network structured into layers is deep if it comprises more
than one hidden layer.

\begin{figure}
  \centering
  \includegraphics[width=0.5\textwidth]{figures/neural-network.pdf}
  \caption{Feed forward artificial neural network \cite{feed-forward-img}}
  \label{fig:intro:neural-network}
\end{figure}

Training algorithms, often called neural network optimizers, try to minimize
the value of the loss function on training data. The loss function measures how
well the network performs in each individual training sample, for example, the
difference between truth and prediction.

Neural networks are usually trained with a variation of stochastic gradient
descent (SGD) or a derived algorithm \cite[p.~149]{goodfellow2016deep}. SGD
repeatedly and randomly selects several training samples called a mini-batch
and computes the loss function gradient with respect to trained parameters
(i.e. weights and biases) with the back propagation algorithm
\cite[p.~149]{goodfellow2016deep}. Backpropagation is an algorithm which
computes the gradient of a loss function with respect to network parameters
with application of the chain rule \cite[p.~201]{goodfellow2016deep}. A
multiple, which is called the learning rate, of the calculated gradient is then
subtracted from the parameters \cite[p.~150]{goodfellow2016deep}.

A common pattern in the architecture of binary classification neural networks
is to have a single output neuron, with an appropriate activation function and
a discrimination threshold on output neuron activation.

Convolutional neural networks (CNNs) are a class of neural networks that
contain one or more convolution layers (see Figure \ref{fig:intro:cnn}).
Convolution layers contain neurons with shared weights and a limited perceptive
field; the weights of the neurons are shift invariant
\cite[p.~326]{goodfellow2016deep}. Thanks to its features, CNNs greatly reduce
a number of learnable parameters and are therefore easier to train and less
prone to overfitting \cite[p.~339]{goodfellow2016deep}.

\begin{figure}
  \centering
  \includegraphics[width=0.4\textwidth]{figures/cnn.png}
  \caption{Convolutional layer (top) over preceding layer (bottom)
    \cite{dumoulin2016guide}}
  \label{fig:intro:cnn}
\end{figure}

Formula \ref{eq:intro:cnn} provides the activation of neurons in a 2D
convolutional layer, where $y_{i, j}$ is the activation of the neuron at
position $(i, j)$ in a 2D grid of neurons, $g: \mathbb{R} \rightarrow
\mathbb{R}$ is the activation function, $k_{m, n} \in \mathbb{R}$ is a shared
weight from kernel $k \in \mathbb{R}^{M \times N}$, $x_{i + m, j + n}$ is the
output of neuron $(i + m, j + n)$ from the preceding 2D layer, and $b \in
\mathbb{R}$ is the shared bias \cite[p.~328]{goodfellow2016deep}.

\begin{equation}
  y_{i, j} = g\left(\sum_m \sum_n k_{m, n} \cdot x_{i + m, j + n} - b\right)
  \label{eq:intro:cnn}
\end{equation}

Recurrent neural networks (RNNs) are a family of neural networks where the
connections between nodes form a directed graph along a sequence
\cite[p.~368]{goodfellow2016deep}. In other words, the computational graph of
these networks contain loops that unfold over a time variable. This
architecture allows for the processing of possibly arbitrarily long sequences
of data \cite[p.~367]{goodfellow2016deep}.

\section{\label{ch:background:evaluation}Evaluation}

The four following basic metrics can be estimated for any binary classification
algorithm with respect to truth data:

\begin{itemize}
  \item true positive rate ($\mathit{TPR}$) -- probability of a positive sample
    being classified as positive by the classification algorithm,
  \item true negative rate ($\mathit{TNR}$) -- probability of a negative sample
    being classified as negative,
  \item false positive rate ($\mathit{FPR}$) -- probability of a negative
    sample being classified as positive,
  \item false negative rate ($\mathit{FNR}$) -- probability of a positive
    sample being classified as negative.
\end{itemize}

These constants are estimated with formulas $\mathit{TPR} =
\frac{\mathit{TP}}{P}$, $\mathit{TNR} = \frac{\mathit{TN}}{N}$, $\mathit{FPR} =
\frac{\mathit{FP}}{N}$ and $\mathit{FNR} = \frac{\mathit{FN}}{P}$, where
$\mathit{TP}$, $\mathit{TN}$, $\mathit{FP}$, $\mathit{FN}$, $P$, and $N$ are
the number of true positives, true negatives, false positive, false negatives,
the number of positive samples, and the number of negative samples,
respectively, and obtained by running the algorithm on a test dataset.

Many other metrics could be derived from these constants.

\subsection{Accuracy}

The accuracy of an algorithm is the probability of a sample being classified
correctly, given by Formula \ref{eq:intro:accuracy}, where $p$ is the prior
probability of the positive class.

\begin{equation}
  \mathit{Accuracy} = \mathit{TPR} \cdot p + \mathit{TNR} \cdot (1 - p)
  \label{eq:intro:accuracy}
\end{equation}

This metric is of limited use in case of highly unbalanced data. For example,
an algorithm classifying all samples as negative would have 99\% accuracy on
data with 99\% of the samples being negative.

\subsection{Precision and Recall}

Precision is a fraction of the true positive classifications in all positive
classifications given by Formula \ref{eq:intro:precision}, where $p$ is the
prior probability of the positive class.

\begin{equation}
  \mathit{Precision} = \frac{\mathit{TPR} \cdot p}{\mathit{TPR} \cdot p +
    \mathit{FPR} \cdot (1 - p)}
  \label{eq:intro:precision}
\end{equation}

Recall is a fraction of the positive samples classified as positive. Recall is
independent of class prior probabilities. See Formula \ref{eq:intro:recall}.

\begin{equation}
  \mathit{Recall} = \frac{\mathit{TPR}}{\mathit{TPR} + \mathit{FNR}} =
  \mathit{TPR}
  \label{eq:intro:recall}
\end{equation}

Both precision and recall might be calculated directly from $\mathit{TP}$,
$\mathit{FP}$, $\mathit{TN}$, $\mathit{FN}$ counts measured on a test dataset.
In such a scenario, calculated precision would differ from
\ref{eq:intro:precision} if a fraction of positive sample was different from
positive class prior $p$.

\subsection{Area Under Curve}

The area under curve (AUC) is an area under a receiver characteristic curve
(ROC). An ROC curve gives dependence of a true positive rate on the false
positive rate obtained on the different threshold settings of a binary
classification algorithm.
