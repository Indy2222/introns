\chapter{\label{ch:data}Data}

\minitoc

Source DNA sequence data and its annotations were downloaded from the Joint
Genome Institute\footnote{https://jgi.doe.gov/fungi}
\cite{grigoriev2014mycocosm}. The data contains FASTA files with the DNA
sequence scaffolds of individual organisms and GFF files with DNA feature
annotations.

As part of this work, all data was uploaded to Google Cloud Storage in a
compressed form, and an automated download and extraction script was created.

\section{\label{ch:data:file-formats}File Formats}

A FASTA file is a text-based representation of DNA sequences. Such a file
contains header lines beginning with character \Verb_>_, followed by an
identifier of the sequence on successive lines. The sequence itself consists of
characters \Verb_ATCGN_ that represent adenine, thymine, cytosine, guanine, and
``any character'' respectively. In the data, each sequence within a FASTA file
represents a scaffold.

The sequences in FASTA files represent scaffolds. A scaffold links together a
non-contiguous series of genomic sequences, comprising sequences separated by
gaps of known length. The linked sequences are typically contiguous sequences
corresponding to read overlaps \cite{ison2013edam}.

The following sample is the first four lines of the FASTA file with the DNA
sequence of the organism Verticillium dahliae.

\begin{Verbatim}[fontsize=\small]
>Supercontig_1.1
AGTATCATGAAGGAAGAACAAGTTGAGGGACATAATTACCTGGGGTGCGGCGCTTACAAGTAAGGGTCGC
TGGGACATCGACCTGGAGGAGGAGAATCATGTAACGCCCCAGCCCGGTCGTCACCAGGACACCAGGCAGG
ACACCCCGCAGGCGATCGGACGCGCCGCACGGACCCACAGGATCACTCACGTGACCGTGACCAGATCACG
\end{Verbatim}

General Feature Format (GFF) is a file with DNA feature annotations. The file
format has three versions. The latest version 3 is used in the data. GFF is a
text-based file in which each line represents a feature annotation. An
annotation consists of nine tab-delimited attributes \cite{gff}:

\begin{itemize}
\item sequence -- name of the sequence, scaffold in our case, where the
  sequence is located,
\item source -- source of the feature, for example, name an institution,
\item feature -- type of the feature, only feature type \Verb_exon_ are used in
  this work,
\item start -- 1-based, inclusive offset of start of the feature,
\item end -- 1-based, inclusive offset of end of the feature,
\item score -- confidence in validity of the feature,
\item strand -- DNA strand where the feature is located. It is one of \Verb_+_,
  \Verb_-_ or \Verb_._ for positive, negative, and undetermined respectively,
\item phase -- phase of CDS features which is always one of 0, 1 or 2,
\item attributes -- additional feature attributes. Format of this field is not
  generally determined. In GFF files used in this work, it contains
  colon-delimited list of attributes, where each is space-delimited attribute
  name and value.
\end{itemize}

The following sample is the first four lines of a GFF file with annotations of
the DNA sequence of the organism Verticillium dahliae.

\begin{Verbatim}[fontsize=\scriptsize]
Supercontig_1   JGI exon    76  572 .   +   .   name "VDBG_00001T0"; transcriptId 224
Supercontig_1   JGI CDS 406 572 .   +   0   name "VDBG_00001T0"; proteinId 1; exonNumber 1
Supercontig_1   JGI start_codon 406 408 .   +   0   name "VDBG_00001T0"
Supercontig_1   JGI exon    631 1621    .   +   .   name "VDBG_00001T0"; transcriptId 224
\end{Verbatim}

\section{\label{ch:data:taxonomy}Taxonomy}

The data consists of eight different phyla, but $89.8\%$ of the organisms (774
of 862) are from Ascomycota and Basidiomycota phyla which make the Dikarya
subkingdom. See Table \ref{table:data:num-organisms} with the number of
organisms per phylum and Figure \ref{fig:data:tree} with a phylogenetic tree of
available fungi.

\begin{table}
  \begin{center}
    \begin{tabular}{ | l | c | }
      \hline
      \textbf{Phylum} & \textbf{Number of Organisms} \\
      \hline
      Ascomycota & 479 \\
      Basidiomycota & 295 \\
      Blastocladiomycota & 4 \\
      Cryptomycota & 1 \\
      Chytridiomycota & 21 \\
      Microsporidia & 8 \\
      Mucoromycota & 38 \\
      Zoopagomycota & 16 \\
      \hline
    \end{tabular}
  \end{center}
  \caption{\label{table:data:num-organisms}Phyla in the dataset}
\end{table}

\begin{figure}
  \centering
  \includegraphics[width=0.8\textwidth]{figures/taxonomy-tree.png}
  \caption{Phylogenetic tree of fungi with sequenced genomes as displayed on
    the website of Joint Genome Institute \cite{grigoriev2014mycocosm}}
  \label{fig:data:tree}
\end{figure}

\section{\label{ch:data:stats}Data Statistics}

\begin{table}
  \begin{center}
    \begin{tabular}{ | l | c | }
      \hline
      \textbf{Taxonomy Rank} & \textbf{Number of Taxa} \\
      \hline
      Phylum & 8 \\
      Class & 46 \\
      Order & 124 \\
      Family & 307 \\
      Genus & 566 \\
      Species & 862 \\
      \hline
    \end{tabular}
  \end{center}
  \caption{Number of taxa on different taxonomy ranks}
\end{table}

In total, 16\,067\,492 introns, distributed over 5\,557\,272 individual genes,
were identified from exon annotations on all 862 organisms in the dataset. This
number includes only introns inside coding sequences, see Section
\ref{ch:automation:features}. Also see Section \ref{ch:automation:overview}
which includes the definition of gene used in this thesis.

The cumulative length of all genes is 8\,396\,757\,084 or 8\,371\,133\,149,
with overlaps counted only once. The cumulative length of all introns is
1\,296\,667\,885 or 1\,293\,833\,629, with counting overlaps only once or
roughly 15\% of genes.

The mean gene length is $1510.9$ nucleotides and the mean intron length is
$80.7$ nucleotides. As much as $93.5\%$ of the introns are 150 nucleotides long
or less with a distinct peek in length distribution around 50 nucleotides. See
Figure \ref{fig:data:intron-len-dist} with intron length distribution and
Figure \ref{fig:data:intron-len-box-plot} that depicts a box plot of intron
lengths on all available phyla.

\begin{figure}
  \centering
  \includegraphics[width=0.8\textwidth]{figures/intron-length-hist.pdf}
  \caption{Distribution of intron length over all organisms in the dataset}
  \label{fig:data:intron-len-dist}
\end{figure}

In this work, only donors and acceptors with consensus dinucleotides were used,
see Section \ref{ch:rcnn:candidates} for more information on this topic.

In total, 15\,792\,942 donors and 15\,828\,432 acceptors were identified in the
data. All occurrences of \Verb_GT_ for donors and \Verb_AG_ for acceptors at
positions different from the positions of true splice sites were considered as
false donors and false acceptors. Moreover, 411\,393\,734 false donors and
498\,273\,594 false acceptors were found inside genes---this gives the
frequency of one false donor per $20.3$ nucleotides and one false acceptor per
$16.8$ nucleotides. False donors are $26.0$ times more frequent than true
donors, and false acceptors are $31.2$ times more frequent than true acceptors
in exon-intron areas.

\begin{figure}
  \centering
  \includegraphics[width=0.8\textwidth]{figures/intron-length-box-plot.pdf}
  \caption{Box plot with intron lengths on all phyla. Whiskers extend to 5\%
    and 95\% percentiles.}
  \label{fig:data:intron-len-box-plot}
\end{figure}
