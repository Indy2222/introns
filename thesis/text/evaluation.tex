\chapter{\label{ch:evaluation}Evaluation}

\minitoc

Detailed evaluation, inspection, and analysis of splice site classification
networks, as well as the performance of the overall intron detection pipeline,
is reported and discussed in this chapter.

Section \ref{ch:evaluation:datasets} gives details of test sample selection and
puts that into the context of other existing research, specific needs, and
goals of this work. Section \ref{ch:evaluation:basic} reports several basic
per-organism evaluation metrics, such as precision and recall, as well as
prediction confidence distributions. Section \ref{ch:evaluation:comparison}
compares the performance of the neural networks with the SVM developed in
\cite{barucic}. Section \ref{ch:evaluation:cpu} describes the CPU usage of the
neural networks and SVM; it also evaluates the possible cost of using the
pipeline in the Google Cloud Platform.

Section \ref{ch:evaluation:sensitivity} analyzes the sensitivity of the neural
networks to single nucleotide mutations/swaps and compares that to known
biological phenomena. Section \ref{ch:evaluation:proximity} reports the error
rate of the networks on negative samples in proximity to a true splice site.
Section \ref{ch:evaluation:lengths} describes the error rate of the network on
introns of various lengths. Section \ref{ch:evaluation:correlation} evaluates
the dependency of donor and acceptor splice site models. And finally, Section
\ref{ch:evaluation:whole} discusses the overall gene prediction pipeline.

\section{\label{ch:evaluation:datasets}Datasets}

Prior to all experiments, the data was split into training, validation, and
test datasets. The performance of splice site classification networks was
evaluated on organisms that were not included in the training and validation
datasets. The test dataset consists of eight organisms, each belonging to a
different phylum. Data from organisms included in training datasets was not
used in any experiment or evaluation before the final evaluation reported in
this chapter. The selected test organisms are the same as in the work
\cite{barucic} to allow for good comparison. See Table
\ref{table:evaluation:test-set} which lists all organisms included in the test
dataset.

\begin{table}
  \begin{center}
    \begin{tabular}{ | l | l | l | }
      \hline
      \textbf{Phylum} & \textbf{Species} & \textbf{Organism ID} \\
      \hline
      Ascomycota & Aspergillus wentii & Aspwe1 \\
      Basidiomycota & Mycena albidolilacea & Mycalb1  \\
      Blastocladiomycota & Allomyces macrogynus & Allma1 \\
      Chytridiomycota & Chytriomyces sp. MP71 & Chytri1 \\
      Cryptomycota & Rozella allomycis & Rozal\_SC1 \\
      Microsporidia & Encephalitozoon hellem & Enche1 \\
      Mucoromycota & Lichtheimia corymbifera & Liccor1 \\
      Zoopagomycota & Coemansia reversa & Coere1 \\
      \hline
    \end{tabular}
  \end{center}
  \caption{\label{table:evaluation:test-set}Organisms included in the test dataset}
\end{table}

The decision to split datasets on the organism lever rather than the sample
level is motivated by the need to test the network ability in order to
generalize previously unseen organisms. The network will be used in a pipeline
executed on metagenomes containing large number of previously unseen organisms.
This sharply contrasts with some studies on automated intron detection because
it usually reports the results measured on organisms included in the training
dataset \cite{zuallaert2018splicerover}.

\section{\label{ch:evaluation:basic}Basic Metrics}

One final neural network architecture was selected for the classification of
both candidate donor splice sites and candidate acceptor splice sites. See
Section \ref{ch:rcnn:architecture}. Two versions of the model with different
input window sizes were selected for the final evaluation and usage. Model
NN\,100 has an input window size of 100 nucleotides and model NN\,400 has an
input window size of 400 nucleotides. The larger model, unsurprisingly, has
better performance but at the cost of roughly $2.9\times$ higher CPU usage per
classification. See Table \ref{table:rcnn:win-size-donor} and Table
\ref{table:rcnn:win-size-acceptor} which compare the performance of the models
with various input window sizes (evaluated on the validation dataset during the
experimentation phase).

The performances of donor splice site classification models are reported in
Table \ref{table:evaluation:donor} and the performances of acceptor splice site
models are reported in Table \ref{table:evaluation:acceptor}
\tablefootnote{Only 20 positive donor samples and 19 positive acceptor samples
  were used during the evaluation of organism Enche1.}. The low precision on
some organisms is largely due to the high ratio of the number of candidate
splice sites to the number of true splice sites. See Section
\ref{ch:data:stats} on data statistics. Also see Section
\ref{ch:rcnn:candidates} that describes which splice sites were included in the
datasets.

\begin{table}
  \begin{center}
    \begin{tabular}{ | l | d{2} | d{2} | d{2} | d{2} | d{2} | d{2} | }
      \hline
      & \multicolumn{3}{| l |}{\textbf{NN\,100}} & \multicolumn{3}{| l |}{\textbf{NN\,400}} \\
      \hline

      & \multicolumn{1}{| l |}{\textbf{Precision}}
      & \multicolumn{1}{| l |}{\textbf{Recall}}
      & \multicolumn{1}{| l |}{\textbf{AUC}}
      & \multicolumn{1}{| l |}{\textbf{Precision}}
      & \multicolumn{1}{| l |}{\textbf{Recall}}
      & \multicolumn{1}{| l |}{\textbf{AUC}} \\

      \hline
      Allma1      & 55.8\% & 74.0\% & 96.1\% & 62.8\% & 82.3\% & 97.2\% \\
      Aspwe1      & 53.3\% & 92.5\% & 98.3\% & 68.1\% & 94.0\% & 98.5\% \\
      Chytri1     & 49.4\% & 71.4\% & 95.5\% & 65.1\% & 85.0\% & 97.7\% \\
      Coere1      &  7.1\% & 61.8\% & 85.5\% &  9.0\% & 64.7\% & 87.8\% \\
      Enche1      &  0.8\% & 100.0\% & 99.4\% & 1.0\% & 95.0\% & 98.2\% \\
      Liccor1     & 56.2\% & 89.9\% & 98.0\% & 75.8\% & 94.8\% & 98.9\% \\
      Mycalb1     & 61.9\% & 78.1\% & 95.2\% & 75.3\% & 82.4\% & 96.6\% \\
      Rozal\_SC1  & 25.6\% & 40.3\% & 86.6\% & 35.7\% & 45.3\% & 90.0\% \\
      \hline
    \end{tabular}
  \end{center}
  \caption{\label{table:evaluation:donor}Performance of donor classification
    models. Model NN\,100 has input window size of 100 nucleotides, starting at
    the splice site and going downstream. Model NN\,400 has input window size
    of 400 nucleotides, spanning exactly 200 nucleotides to both sides from the
    splice site.}
\end{table}

\begin{table}
  \begin{center}
    \begin{tabular}{ | l | d{2} | d{2} | d{2} | d{2} | d{2} | d{2} | }
      \hline
      & \multicolumn{3}{| l |}{\textbf{NN\,100}} & \multicolumn{3}{| l |}{\textbf{NN\,400}} \\
      \hline

      & \multicolumn{1}{| l |}{\textbf{Precision}}
      & \multicolumn{1}{| l |}{\textbf{Recall}}
      & \multicolumn{1}{| l |}{\textbf{AUC}}
      & \multicolumn{1}{| l |}{\textbf{Precision}}
      & \multicolumn{1}{| l |}{\textbf{Recall}}
      & \multicolumn{1}{| l |}{\textbf{AUC}} \\

      \hline
      Allma1      & 70.6\% & 80.2\% & 97.2\% & 66.3\% & 83.7\% & 97.4\% \\
      Aspwe1      & 54.6\% & 93.4\% & 98.5\% & 56.9\% & 93.9\% & 98.6\% \\
      Chytri1     & 59.1\% & 82.0\% & 97.3\% & 60.3\% & 84.9\% & 97.7\% \\
      Coere1      &  5.9\% & 46.6\% & 83.5\% &  7.1\% & 57.0\% & 85.3\% \\
      Enche1      &  0.3\% & 42.1\% & 91.9\% &  0.3\% & 42.1\% & 94.3\% \\
      Liccor1     & 68.5\% & 93.5\% & 98.8\% & 72.3\% & 93.8\% & 98.9\% \\
      Mycalb1     & 65.4\% & 80.8\% & 96.3\% & 63.9\% & 83.0\% & 96.7\% \\
      Rozal\_SC1  & 12.0\% & 22.7\% & 79.6\% & 19.5\% & 30.1\% & 84.1\% \\
      \hline
    \end{tabular}
  \end{center}
  \caption{\label{table:evaluation:acceptor}Performance of acceptor
    classification models. Model NN\,100 has input window size of 100
    nucleotides, starting 100 nucleotides upstream from the splice site and
    going downstream. Model NN\,400 has input window size of 400 nucleotides,
    spanning exactly 200 nucleotides to both sides from the splice site.}
\end{table}

The neural networks output values between 0 and 1. This value needs to be
compared to a threshold to obtain a binary classification. During the
evaluation of all splice site classification models, the threshold was set to
$0.5$. See Figure \ref{fig:evaluation:donor-cdf} and Figure
\ref{fig:evaluation:donor-density} which depict the distribution of the
prediction values on both positive and negative samples of the donor model.
Figure \ref{fig:evaluation:acceptor-cdf} and Figure
\ref{fig:evaluation:acceptor-density} visualize the same properties on the
acceptor model.

The confidence intervals visualized in Figure
\ref{fig:evaluation:donor-density} and Figure
\ref{fig:evaluation:acceptor-density} are calculated with Formula
\ref{eq:evaluation:ci-low} and Formula \ref{eq:evaluation:ci-high} derived in
\cite[p.~176]{johnson2005univariate}.

\begin{equation}
  \theta_L = 0.5 \chi^2_{2k, \alpha / 2}
  \label{eq:evaluation:ci-low}
\end{equation}

\begin{equation}
  \theta_U = 0.5 \chi^2_{2k + 2, 1 - \alpha / 2}
  \label{eq:evaluation:ci-high}
\end{equation}

Kolmogorov–Smirnov statistic computed on the cumulative distribution functions
of the prediction values on positive and negative samples is $0.83$ at $0.18$
for the donor model and $0.79$ at $0.18$ for the acceptor model. The values
imply the high ability of the model to discriminate positive and negative
samples at the threshold point of $0.18$. During the evaluation and production
use of the models, a higher threshold of $0.5$ was used to compensate for the
high positive to negative sample rate ratio.

The prediction distributions were measured over samples from all organisms
included in the test dataset. In total, 10\,000 positive samples and 10\,000
negative samples were included from each organism, except those which did not
have enough annotated features. The only organism which was largely
underrepresented was Enche1. Prediction distribution on different organisms,
especially organisms from different phyla, differ. Likewise, the used sample
set does not represent true organism distribution in nature or in extracted
metagenomes. This implies the need for a careful interpretation of the plots.

\begin{figure}
  \centering
  \includegraphics[width=\textwidth]{figures/donor_evaluation/cdf.pdf}
  \caption{Cumulative distribution function with Kolmogorov–Smirnov statistic
    of prediction values from NN\,400 model on positive and negative samples of
    donor splice sites. }
  \label{fig:evaluation:donor-cdf}
\end{figure}

\begin{figure}
  \centering
  \includegraphics[width=\textwidth]{figures/donor_evaluation/density.pdf}
  \caption{Density of prediction values from NN\,400 model on positive and
    negative samples of donor splice sites.}
  \label{fig:evaluation:donor-density}
\end{figure}

\begin{figure}
  \centering
  \includegraphics[width=\textwidth]{figures/acceptor_evaluation/cdf.pdf}
  \caption{Cumulative distribution function with Kolmogorov–Smirnov statistic
    of prediction values from NN\,400 model on positive and negative samples of
    acceptor splice sites.}
  \label{fig:evaluation:acceptor-cdf}
\end{figure}

\begin{figure}
  \centering
  \includegraphics[width=\textwidth]{figures/acceptor_evaluation/density.pdf}
  \caption{Density of prediction values from NN\,400 model on positive and
    negative samples of acceptor splice sites.}
  \label{fig:evaluation:acceptor-density}
\end{figure}

\section{\label{ch:evaluation:comparison}Comparison with SVM}

Table \ref{table:evaluation:comp-svm} compares the true positive rate and true
negative rate of two neural networks with different input window sizes and SVM,
as initially developed in \cite{barucic}. The smaller neural network (NN\,100)
had an input window of 100 nucleotides going from the splice site in the intron
direction (downstream in the case of a donor and upstream in the case of an
acceptor). The larger neural network (NN\,400) had a window size of 400
nucleotides and spanned 200 nucleotides to both directions from the splice
site. The SVM used for the evaluation was trained solely on data from the
Basidiomycota phylum.

Both neural networks produce much less false positive predictions compared to
the SVM. Furthermore, the NN\,400 model outperforms the SVM in every
measurement, except for $\mathit{TPR}$ on donors in the Basidiomycota phylum.

The improvement in $\mathit{TNR}$ achieved with the neural networks is very
important due to the large ratio of the false candidate splice sites to the
true candidate splice sites.

All classification methods in Table \ref{table:evaluation:comp-svm} were
evaluated on the same dataset. Negative samples were taken from genetic regions
defined as the area between the beginning of the first exon of the gene and the
end of the last exon of the gene. These regions are wider than the CDS and
intra-CDS regions used in other parts of this work, see Section
\ref{ch:automation:overview}.

\begin{table}
  \begin{center}
    \begin{tabular}{ | l | l | d{2} | d{2} | d{2} | d{2} | d{2} | d{2} | }
      \hline

      & &
      \multicolumn{3}{| l |}{\textbf{Ascomycota}} &
      \multicolumn{3}{| l |}{\textbf{Basidiomycota}} \\

      \hline

      & &
      \multicolumn{1}{| l |}{\textbf{SVM}} &
      \multicolumn{1}{| l |}{\textbf{NN\,100}} &
      \multicolumn{1}{| l |}{\textbf{NN\,400}} &
      \multicolumn{1}{| l |}{\textbf{SVM}} &
      \multicolumn{1}{| l |}{\textbf{NN\,100}} &
      \multicolumn{1}{| l |}{\textbf{NN\,400}} \\

      \hline

      \multirow{2}{*}{\textbf{Donor}}
      & $\mathit{TPR}$ & 86.6\% & 84.9\% & 87.2\% & 91.6\% & 86.3\% & 89.9\% \\
      & $\mathit{TNR}$ & 94.9\% & 97.2\% & 97.7\% & 95.9\% & 97.6\% & 98.1\% \\

      \hline

      \multirow{2}{*}{\textbf{Acceptor}}
      & $\mathit{TPR}$ & 83.4\% & 85.3\% & 86.4\% & 88.1\% & 87.7\% & 88.8\% \\
      & $\mathit{TNR}$ & 93.8\% & 97.7\% & 97.9\% & 93.8\% & 97.8\% & 97.7\% \\

      \hline
    \end{tabular}
  \end{center}
  \caption{\label{table:evaluation:comp-svm}Comparison of true positive rate
    ($\mathit{TPR}$) and true negative rate ($\mathit{TNR}$) between SVM,
    neural network with window size 100 (NN\,100) and neural network with
    window size 400 (NN\,400).}
\end{table}

\section{\label{ch:evaluation:cpu}Computational Intensity}

The NN\,400 model uses $2.88\times$ more CPU time per inference than the
NN\,100 model. The SVM uses $46.4\times$ more CPU time per inference than the
NN\,100 model. This makes the use of the neural networks more practical on
large metagenomes.

The NN\,100 model uses 1.37 CPU seconds per 1000 predictions, while the NN\,400
model uses 3.94 CPU seconds per 1000 predictions. The measurement was performed
on the laptop ThinkPad T490 with Intel\textsuperscript{\tiny\textregistered}
Core™ i5-8265U CPU @ 1.60GHz. The measurement was performed with all samples
loaded in memory. TensorFlow version 2.3.1 compiled with instructions
\Verb_AVX2_ and \Verb_FMA_ enabled was used. The measurement included only the
CPU time usage of Python call \Verb_model.predict(inputs)_.

There is around $10^8$ donor and acceptor candidate splice sites in a
metagenome of size of $10^9$ nucleotides. Using the NN\,100 model, all
candidate splice sites in this hypothetical dataset could be classified in
137\,000 CPU seconds or 9.5 hours on four CPU cores. Both CPU and real time
would be much smaller on a server CPU, GPU or TPU.

One hour of a single CPU on an E2 machine type in Google Cloud Platform costs
0.021811\,USD \cite{gcp-pricing}. Using this price and ignoring the differences
between CPUs, any inefficiencies, or overhead, all introns in the
aforementioned hypothetical metagenome could be classified for 0.83\,USD.

\section{\label{ch:evaluation:sensitivity}Positional Sensitivity}

Figure \ref{fig:evaluation:donor-divergence-all} visualizes an estimation of
the Kullback–Leibler divergence between the distributions of donor model
inferences made on unmodified sequences and sequences with single nucleotide
symbol swaps.

\begin{figure}
  \centering
  \includegraphics[width=\textwidth]{figures/donor_evaluation/divergence-all.pdf}
  \caption{Kullback–Leibler divergences between NN\,400 donor model inferences
    on unmodified input sequences and inferences on sequences with single
    nucleotide modifications at various positions. Position-wise maximum over
    swaps to adenine, thymine, cytosine, adenine are shown.}
  \label{fig:evaluation:donor-divergence-all}
\end{figure}

The figure illustrates that the network is highly sensitive in the close
vicinity of the splice site with decreasing sensitivity in the downstream
(intron) direction. Almost no sensitivity could be found to one nucleotide
swaps 9 nucleotides and further upstream from the splice site. Part of this
sensitivity spike around the splice site is consistent with the literature
which reports that the 5\textquotesingle{}-terminus of the U1 snRNA component
of spliceosome binds to the nearly perfect Watson-Crick complement sequence
\Verb_CAGGURAGU_ that spans -3 to +6 around 5\textquotesingle{}-end of an
intron \cite{de2013exon}.

The second peak of sensitivity is found around 40 nucleotides downstream from
the splice site. There is a high frequency of introns of a length between 40
and 75 nucleotides, see Figure \ref{fig:data:intron-len-dist}. The branch point
is located 18 to 40 nucleotides upstream from the acceptor splice site
\cite{clancy2008rna}. The relative location of the second peak likely implies
that the network is learned to recognize area around branch point. This is
further supported by Figure \ref{fig:evaluation:donor-divergence-60}, where
positive samples were limited to the splice sites of introns of a length of 60
nucleotides. Figure \ref{fig:evaluation:donor-divergence-60} also displays a
sensitivity peak at the location of the acceptor splice site.

\begin{figure}
  \centering
  \includegraphics[width=\textwidth]{figures/donor_evaluation/divergence-60.pdf}
  \caption{Kullback–Leibler divergences between NN\,400 donor model inferences
    on unmodified input sequences and inferences on sequences with single
    nucleotide modifications at various positions. Position-wise maximum over
    swaps to adenine, thymine, cytosine, adenine are shown. Positive samples
    are limited to the splice site of introns of a length 60 nucleotides.}
  \label{fig:evaluation:donor-divergence-60}
\end{figure}

Figure \ref{fig:evaluation:acceptor-divergence-all} visualizes an estimation of
the Kullback–Leibler divergence between the distributions of acceptor model
inferences made on unmodified sequences and sequences with single nucleotide
symbol swaps. Figure \ref{fig:evaluation:acceptor-divergence-60} visualizes the
divergence on introns of a length of 60 nucleotides. The acceptor sensitivity
plot largely overlaps with the donor sensitivity plot, with a notable
difference of the sensitivity decreasing more steeply near the opposite splice
site and vice versa. Compared to the donor model, the acceptor model is more
sensitive to one nucleotide swap in the negative samples.

\begin{figure}
  \centering
  \includegraphics[width=\textwidth]{figures/acceptor_evaluation/divergence-all.pdf}
  \caption{Kullback–Leibler divergences between NN\,400 acceptor model
    inferences on unmodified input sequences and inferences on sequences with
    single nucleotide modifications at various positions. Position-wise maximum
    over swaps to adenine, thymine, cytosine, adenine are shown.}
  \label{fig:evaluation:acceptor-divergence-all}
\end{figure}

\begin{figure}
  \centering
  \includegraphics[width=\textwidth]{figures/acceptor_evaluation/divergence-60.pdf}
  \caption{Kullback–Leibler divergences between NN\,400 acceptor model
    inferences on unmodified input sequences and inferences on sequences with
    single nucleotide modifications at various positions. Position-wise maximum
    over swaps to adenine, thymine, cytosine, adenine are shown. Positive
    samples are limited to the splice site of introns of a length of 60
    nucleotides.}
  \label{fig:evaluation:acceptor-divergence-60}
\end{figure}

A similar analysis based on different techniques was done in
\cite{zuallaert2018splicerover} which used a CNN trained and evaluated on
Arabidopsis and human. Visualizations reported in that paper, however, differ
with the results reported in this section, especially in the areas more than 10
nucleotides distant from the splice sites.

Kullback–Leibler divergence estimation is calculated with Formula
\ref{ch:evaluation:divergence} on two $n$-tuples of samples i.i.d. drawn from
distributions $p$ and $q$ respectively. $\nu_k(i)$ is distance of the $i$-th
sample from the first $n$-tuple to the $k$-th nearest neighbor from the second
$n$-tuple; $\rho_k(i)$ is distance of the $i$-th sample from the first
$n$-tuple to the $k + 1$ nearest neighbor from the same $n$-tuple. This
equation was derived from \cite{wang2006nearest}, $n = 1000$ and $k = 10$ were
used.

\begin{equation}
  D_{n}(p \parallel q) = \frac{1}{n} \sum_{i = 1}^n log
  \frac{\nu_k(i)}{\rho_k(i)} + \log \frac{n}{n - 1}
  \label{ch:evaluation:divergence}
\end{equation}

\section{\label{ch:evaluation:proximity}False Positives in the Proximity to a Splice Site}

Figure \ref{fig:evaluation:donor-splice-site-dist-wide} shows the dependence of
the donor model performance on negative splice site examples to the distance to
the closest real donor splice site. Figure
\ref{fig:evaluation:donor-splice-site-dist-narrow} displays this dependence
only in a narrow neighborhood of true splice sites.

The plots display the mean predicted value and the false positive rate for each
distance bin. The mean predicted value is equal to the mean prediction error
because only negative samples are used. The measurements are done on bins of
size 5 in Figure \ref{fig:evaluation:donor-splice-site-dist-wide} and on bins
of size 3 in Figure \ref{fig:evaluation:donor-splice-site-dist-narrow}. The
figures differ due to the overall captured distance range and different bin
sizes.

This data was generated on the validation dataset because there were not enough
samples in the test dataset to calculate the unbiased report without too much
noise. Only splice sites with consensus dinucleotides were included among the
negative samples. However, the closest true splice site was selected from the
set of all splice sites.

\begin{figure}
  \centering
  \includegraphics[width=\textwidth]{figures/donor_evaluation/splice-site-dist-wide.pdf}
  \caption{NN\,400 donor model error rate dependence on the relative position
    of the nearest true donor splice site.}
  \label{fig:evaluation:donor-splice-site-dist-wide}
\end{figure}

\begin{figure}
  \centering
  \includegraphics[width=\textwidth]{figures/donor_evaluation/splice-site-dist-narrow.pdf}
  \caption{NN\,400 donor model error rate dependence on relative position of
    the nearest true donor splice site.}
  \label{fig:evaluation:donor-splice-site-dist-narrow}
\end{figure}

Figure \ref{fig:evaluation:acceptor-splice-site-dist-wide} and Figure
\ref{fig:evaluation:acceptor-splice-site-dist-narrow} visualize the same
dependency for the acceptor splice site model. An interesting difference from
the donor model is that the error spike around the true splice site is higher
and wider.

\begin{figure}
  \centering
  \includegraphics[width=\textwidth]{figures/acceptor_evaluation/splice-site-dist-wide.pdf}
  \caption{NN\,400 acceptor model error rate dependence on the relative
    position of the nearest true donor splice site.}
  \label{fig:evaluation:acceptor-splice-site-dist-wide}
\end{figure}

\begin{figure}
  \centering
  \includegraphics[width=\textwidth]{figures/acceptor_evaluation/splice-site-dist-narrow.pdf}
  \caption{NN\,400 acceptor model error rate dependence on the relative
    position of the nearest true donor splice site.}
  \label{fig:evaluation:acceptor-splice-site-dist-narrow}
\end{figure}

The error spike around true splice sites leads to the higher than uniform
presence of false intron detections with an almost perfect overlap with the
true introns compared to introns with a low overlap with the true introns.

After the detected splice sites are combined into whole introns, among the
overlapping intron detections only those detected with the largest splice site
prediction values are kept. It is expected that this would lead to the
selection of the true introns in the majority of the cases.

Likewise, a small negative effect on gene prediction from incorrect intron
detections which large relative overlap with the true introns is expected.

\section{\label{ch:evaluation:lengths}Introns of Various Lengths}

Figure \ref{fig:evaluation:donor-intron-length-error} and Figure
\ref{fig:evaluation:acceptor-intron-length-error} show the dependency of the
error rates of donor and acceptor models, respectively, on the splice sites
associated with introns of various lengths. The figures demonstrate that the
models are systematically not recognizing the splice sites of introns shorter
than 40 nucleotides. The error rate is the smallest in the area around 55
nucleotides and goes up for splice sites of longer introns. Low performance on
short introns is more pronounced in the acceptor model.

The effect could be, in part, explained by the distribution of intron lengths
in the data, see Figure \ref{fig:data:intron-len-dist}. Training samples of the
models were randomly drawn from the dataset of training organisms. Therefore,
the lengths of most of the introns associated with the positive splice sites
the model ``saw'' during training were concentrated around 55 nucleotides.

Introns having lengths shorter than 30 nucleotides are extremely rare in the
gene databases across all eukaryotic organisms
\cite{piovesan2015identification}. Their existence in the data might be a
result of incorrect annotation, and it is hypothesized that they do not exist
in nature due to the minimum sequence elements needed for their splicing
\cite{piovesan2015identification}. Introns shorter than 30 nucleotides are
rare, but they are present in the data used in this work. In light of this, the
high error rate on the splice sites of very short introns could be caused by
data quality.

\begin{figure}
  \centering
  \includegraphics[width=\textwidth]{figures/donor_evaluation/intron-lenght-error.pdf}
  \caption{Dependence of the NN\,400 donor model error rate on positive splice
    site samples and the lengths of associated introns.}
  \label{fig:evaluation:donor-intron-length-error}
\end{figure}

\begin{figure}
  \centering
  \includegraphics[width=\textwidth]{figures/acceptor_evaluation/intron-lenght-error.pdf}
  \caption{Dependence of NN\,400 acceptor model error rate on positive splice
    site samples and lengths of associated introns.}
  \label{fig:evaluation:acceptor-intron-length-error}
\end{figure}

\section{\label{ch:evaluation:correlation}Donor and Acceptor Model Correlation}

Figure \ref{fig:evaluation:donor-acceptor-dependency} depicts the output
dependency of the donor splice site model and the acceptor splice site model
when predictions are performed on the (opposite) splice sites of the same
intron. Per organisms correlations of the outputs are given in Table
\ref{table::evaluation:donor-acceptor-dependency}.

The correlation is negative only for Rozal\_SC1 and Enche1; it is larger than
$0.6$ for all other test organisms. This might be related to the fact that the
performance on the organisms Rozal\_SC1 and Enche1 is lower by a big margin
compared to all the other test organisms, see Table
\ref{table:evaluation:donor} and Table \ref{table:evaluation:acceptor}.

\begin{table}
  \begin{center}
    \begin{tabular}{ | l | d{2} | }
      \hline

      \multicolumn{1}{| l |}{\textbf{Organism ID}} &
      \multicolumn{1}{| l |}{\textbf{Correlation}} \\

      \hline
      Aspwe1 & 0.69 \\
      Mycalb1 & 0.63 \\
      Allma1 & 0.73 \\
      Chytri1 & 0.65 \\
      Rozal\_SC1 & -0.05 \\
      Enche1 & -0.13 \\
      Liccor1 & 0.64 \\
      Coere1 & 0.66 \\
      \hline
    \end{tabular}
  \end{center}
  \caption{\label{table::evaluation:donor-acceptor-dependency}The Pearson
    correlation coefficient of NN\,400 donor and NN\,400 acceptor model outputs
    on the splice sites of the same intron.}
\end{table}

\begin{figure}
  \centering
  \includegraphics[width=\textwidth]{figures/donor-acceptor-dependency.pdf}
  \caption{Dependency of NN\,400 donor and NN\,400 acceptor model outputs on
    the splice sites of the same intron.}
  \label{fig:evaluation:donor-acceptor-dependency}
\end{figure}

The data illustrates that an intron with a ``hard-to-recognize'' splice site is
of diminished ``visibility'' even for the model of the opposite splice site. In
Section \ref{ch:evaluation:sensitivity}, it is demonstrated that both the donor
splice site model and the acceptor splice site model are sensitive in the
region of the opposite splice site---this likely explains the correlations.

If donor and acceptor models were independent, the probability of the detection
of an intron would be a multiplication of the probabilities of its splice sites
being detected (independently). With strong correlation of donor and acceptor
model outputs, the overall likelihood of the detection of a true intron is much
higher. This is shown by Table \ref{table::evaluation:both-splice-sites} that
gives a percentage of introns with both splice sites detected by the real
models and by hypothetical statistically independent models.

\begin{table}
  \begin{center}
    \begin{tabular}{ | l | d{2} | d{2} | }
      \hline

      \multicolumn{1}{| l |}{\textbf{Organism ID}} &
      \multicolumn{1}{| l |}{\textbf{Detected Introns}} &
      \multicolumn{1}{| l |}{\textbf{Multiplication}} \\

      \hline
      Aspwe1     & 91.6\% & 88.0\% \\
      Mycalb1    & 72.8\% & 65.2\% \\
      Allma1     & 78.0\% & 69.8\% \\
      Chytri1    & 79.2\% & 72.7\% \\
      Rozal\_SC1 & 12.5\% & 13.6\% \\
      Enche1     & 42.1\% & 42.1\% \\
      Liccor1    & 92.0\% & 89.2\% \\
      Coere1     & 51.1\% & 37.2\% \\
      \hline
    \end{tabular}
  \end{center}
  \caption{\label{table::evaluation:both-splice-sites}The percentage of introns
    whose splice sites were recognized by both the NN\,400 donor model and the
    NN\,400 acceptor model and the percentage of detected introns, if the
    models were statistically independent but with the same true positive
    rate.}
\end{table}

\section{\label{ch:evaluation:whole}Whole Gene Prediction Pipeline}

As discussed in Chapter \ref{ch:intro} and Chapter \ref{ch:motivation}, the
primary purpose of splice site detection models and their combination to the
full intron detection pipeline is to improve the performance of a larger gene
prediction pipeline, which is utilized to detect gene homologies in novel and
known fungal DNA sequences in metagenomes.

Utilization of the intron detection pipeline based on SVM, as developed in
\cite{barucic}, has significantly improved the gene prediction by increasing
the number of discovered gene homologies when the gene prediction pipeline was
executed separately with and without the intron detection.

Based on the results with SVM models and better performance of the neural
networks compared to SVM, it is expected that the gene prediction will further
benefit with the introduction of the neural network-based intron detection
pipeline.
