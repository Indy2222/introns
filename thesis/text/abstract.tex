\begin{minipage}[t]{0.48\textwidth}
  This work is concerned with the detection of introns in metagenomes with deep
  neural networks. Exact biological mechanisms of intron recognition and
  splicing are not fully known yet and their automated detection has remained
  unresolved.

  Detection and removal of introns from DNA sequences is important for the
  identification of genes in metagenomes and for searching for homologs among
  the known DNA sequences available in public databases. Gene prediction and
  the discovery of their homologs allows the identification of known and new
  species and their taxonomic classification.

  Two neural network models were developed as part of this thesis. The models'
  aim is the detection of intron starts and ends with the so-called donor and
  acceptor splice sites. The splice sites are later combined into candidate
  introns which are further filtered by a simple score-based overlap resolving
  algorithm.

  The work relates to an existing solution based on support vector machines
  (SVM). The resulting neural networks achieve better results than SVM and
  require more than order of magnitude less computational resources in order to
  process equally large genome.

  \vspace{\baselineskip}
  \textbf{Keywords:} fungi, fungal genomes, neural networks, itron detection,
  metagenome
\end{minipage}
\hspace{0.04\textwidth}
\begin{minipage}[t]{0.48\textwidth}
  Tato práce se zabývá detekcí intronů v metagenomech hub pomocí hlubokých
  neuronových sítí. Přesné biologické mechanizmy rozpoznávání a vyřezávání
  intronů nejsou zatím plně známy a jejich strojová detekce není považovaná za
  vyřešený problém.

  Rozpoznávání a vyřezávání intronů z DNA sekvencí je důležité pro identifikaci
  genů v metagenomech a hledání jejich homologií mezi známými DNA sekvencemi,
  které jsou dostupné ve veřejných databázích. Rozpoznání genů a nalezení
  jejich případných homologů umožňuje identifikaci jak již známých tak i nových
  druhů a jejich taxonomické zařazení.

  V rámci práce vznikly dva modely neuronových sítí, které detekují začátky a
  konce intronů, takzvaná donorová a akceptorová místa sestřihu. Detekovaná
  místa sestřihu jsou následně zkombinována do kandidátních intronů.
  Překrývající se kandidátní introny jsou poté odstraněny pomocí jednoduchého
  skórovacího algoritmu.

  Práce navazuje na existující řešení, které využívá metody podpůrných vektorů
  (SVM). Výsledné neuronové sítě dosahují lepších výsledků než SVM a to při
  více než desetinásobně nižším výpočetním čase na zpracování stejně obsáhlého
  genomu.

  \vspace{\baselineskip}
  \textbf{Klíčová slova:} genom hub, neuronové síťe, detekce intronů, metagenom
\end{minipage}
