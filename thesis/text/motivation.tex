\chapter{\label{ch:motivation}Motivation}

\minitoc

DNA sequencing costs are sharply decreasing \cite{NHGRI_sequencing_costs}.
GenBank is a genetic sequence database of all publicly available DNA sequences
\cite{benson2012genbank}. The number of bases in the GenBank database has
doubled approximately every 18 months from 1982 to present (2019), and its
database consists of over 366 billion bases \cite{genbank_release_notes}. The
consequence of ever-cheaper DNA sequencing and exponential sequence database
growth is the need for an increased capacity for DNA data handling and
analysis. An automated and scalable solution to DNA annotation would partially
fulfill this need.

Fungal and bacterial organisms play an important role in various ecosystems
including the floor and soil of forests
\cite{christensen2005wood}\cite{bani2018role}. Some fungal species are capable
of decomposing cellulose and various biopolymers and are involved in the
decomposition of deadwood and litter. The decomposition goes in stages, and in
each stage different organisms with varying diversity contribute to the process
\cite{bani2018role}. Nontrivial dependencies of different fungal organisms and
decomposition stages were identified and a lot of the relationships are yet to
be known \cite{bani2018role}. Decomposed wood and litter is important for some
plant species \cite{bani2018role}. The presence of wood inhabiting fungi might
be a good indicator of overall forest biodiversity \cite{christensen2005wood}.

Not only for the previously mentioned reasons, the ability to identify fungal
species in metagenomes recovered from soil and other environmental samples is a
potent tool in the biological study of forests and other ecosystems. The task
of fungal species identification in a metagenome is approached by searching for
homologous genes in the DNA sequences by applying sequence alignment algorithms
to all known protein and gene sequence databases. The search for homologs does
not only allow the identification of known species, but also searches for
novel, evolutionary related species.

To achieve good search results, we detect and remove introns in long DNA
sequence scaffolds before the search. This should improve protein sequence
match scores because introns are spliced before RNA to protein translation. A
reason for the improvement is that the match score of alignment algorithms for
homologous DNA sequences is the highest in most preserved parts of such DNA
sequences---i.e. exons. The functional regions of DNA sequences tend to be more
highly conserved than the remaining parts of the DNA
\cite{morgenstern2002exon}. See Section \ref{ch:background:alignment} for more
information on sequence alignment.
