\chapter{\label{ch:intro}Introduction}

\minitoc

The kingdom of fungi is largely undiscovered and comprises 2.2--3.8 million
species \cite{hawksworth2017fungal}. Only a small fraction of this---around
100\,000 species---has been described in the literature
\cite{hawksworth2012global}. The kingdom plays a significant role in public
health, food biosecurity, and biodiversity \cite{microbiology2017stop}. It is,
therefore, important to develop new, computer-aided methods to bridge the gap
between what is known and what is unknown.

A gene is a sequence of DNA nucleotides encoding a gene product. A gene product
is a protein or form of RNA. In most organisms, the DNA or RNA sequence that
encodes the product is not continuous but interleaved with introns, which are
spliced out during gene expression. The remaining parts are called exons
\cite{alberts2018molecular}.

Metagenomics is the study of combined genetic material as retrieved directly
from a natural environment. The collection of genetic material retrieved from a
sample is known as metagenome, which, as opposed to a single genome, consist of
the DNA sequences of multiple organisms.

Gene homology refers to the similarity between the genetic sequences of two
genes due to shared ancestry. Two genes with similar DNA sequences and shared
ancestry are called homologous. The identification of homologous genes can be
used in recognizing individual organisms or kinships in the metagenome.

In a study, forest soil samples were extracted and their genetic material
sequenced. Traditional sequence alignment methods were utilized for gene
prediction within the metagenome, leading to the identification of a number of
fungal species within the sample. But the sample was significantly smaller than
expectations based on the estimates of a number of existing fungal species. The
present work seeks to improve the results of the gene prediction procedure. For
this, it identifies and excises introns from the metagenome and keeps only
evolutionarily more preserved exons in the sequences.

The goal of this work is to create a multistep software pipeline that
identifies introns in DNA sequences of potentially unknown fungal organisms.
The use of deep neural networks is elaborated. Their performances and
computational resource usage are compared with previously used methods based on
support vector machines and the potential of using the results in a larger gene
prediction pipeline.

The work uses a large preexisting annotated fungal DNA dataset. The solution is
subdivided into the following sub-goals:

\begin{itemize}
  \item detection of candidate splice site positions with the search for
    consensus dinucleotides,
  \item classification of candidate intron start positions, i.e. donor splice
    sites,
  \item classification of candidate intron end positions, i.e. acceptor splice
    sites,
  \item combining detected splice sites into intron start-end pairs based on
    their mutual distance,
  \item filtering the intron candidates resulting from the previous step by an
    overlap-resolving algorithm.
\end{itemize}

\section{Text Structure}

This chapter (\ref{ch:intro}) introduces the work by briefly explaining several
important biological concepts and puts the work into its context. Chapter
\ref{ch:background} explains biological phenomena related to the work, some
concepts and techniques from computer science--basics of recurrent and
convolutional neural networks (RCNN) and statistical measures used for
evaluation. Chapter \ref{ch:motivation} talks about further motivation for this
work, and Chapter \ref{ch:current-research} summarizes current research on the
topic. Chapter \ref{ch:data} contains an overview of the used data, the
statistics computed on the data, their source, and the used file types.

Chapter \ref{ch:rcnn} describes the recurrent convolutional networks used for
splice site classification. It also describes the architecture and training of
the networks with theoretical and practical reasoning. Chapter
\ref{ch:automation} talks about how data pre-processing and RCNN training and
evaluation were automated for fast and reproducible experimentation. And
finally, Chapter \ref{ch:evaluation} talks about the evaluation and achieved
results.

Chapter \ref{ch:conclusion} concludes the work and outlines possible further
work.
